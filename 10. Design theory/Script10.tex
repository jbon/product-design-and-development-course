\documentclass{article}

% style elements
% Slide numbers {\color{blue}[\stepcounter{slide}Slide \arabic{slide}]}
% Key readings \textsuperscript{\color{Magenta}[key reading]}

\usepackage[utf8]{inputenc}
\usepackage{titling}
\usepackage{hyperref}
\usepackage{color}
\usepackage[dvipsnames]{xcolor}
\usepackage{comment}
\usepackage{textcomp}

\newcounter{slide}
\newcounter{exercise}

\newcommand{\subtitle}[1]{%
  \posttitle{%
    \par\end{center}
    \begin{center}\large#1\end{center}
    \vskip0.5em}%
}

\begin{document}


\title{Design theory}

\author{Product Design and Development (ME30294). Lecture 10. \\ Jérémy Bonvoisin, Dept. Mech. Eng., University of Bath}
\date{Last update: \today}

\maketitle

\begin{abstract}
This lecture introduces some descriptive models of the design activity attempting to picture the essence of design. These descriptive models are put into perspective with more practical aspects of design and are used to explain the relevance of some content delivered in this module. 
\end{abstract}

\tableofcontents

%%%%%%%%%%%%%%%%%%%%%%%%%%%%%%%%%%%%%%%%%%%%%%%%%%%%%%%%%%%%%%%%%%%%%%%%%%%%%%%%%%%%%%%%%%%%%%%%%%%%
%%%%%%%%%%%%%%%%%%%%%%%%%%%%%%%%%%%%%%%%%%%%%%%%%%%%%%%%%%%%%%%%%%%%%%%%%%%%%%%%%%%%%%%%%%%%%%%%%%%%
\section{Prologue: why this lecture?}
\label{sec:prologue}
Along previous lectures, this module strove to provide an overview of good design practices. It aimed at providing methods to build up own's ``designer's toolbox''. It gave tips to create novel design concepts (e.g. lecture 1 and 6), to structure design processes (e.g. lecture 3). It mentioned some of the contextual aspects which are relevant to take into account in today's design practice (e.g. lecture 2 and 9). 

While focusing on how to design, it left out the question of \emph{what design is}. The good reason for that is that there is no commonly acknowledged definition of design. The reasons are partly lexical: the word `design' itself is confusing because it is generally used to depict different meanings as in the sentence ``Design is when designers design a design to produce a design'' \cite{heskett2001past} {\color{blue}[\stepcounter{slide}Slide \arabic{slide}]}. In this sentence, the word `design' is used four times to depict 1) a domain of human activity, 2) an action, 3) an intellectual projection and 4) a physical product---hence giving an idea of the level of confusion. A more fundamental reason why there is no satisfying definition of design is that ``the activity called design is not fully understood'' \cite{hybsEvolutionaryProcessModel1992}. It happens for a great part in the designer's head and offers little interface for observation and experimentation.

While there isn't \emph{one} definition of design, there are many theories shading different lights on what design is. The objective of this lecture is to take a step backward and to review a relevant subset of these theories in order to produce a useful picture of design. These theories provides explanatory frameworks to represent what happens when somebody designs something. They provide a basis to explain some of the oddities of design and justify why some aspects discussed in previous lectures are of relevance. Through this lecture, we hope providing a more thorough understanding of design, hence helping students to make connections between the concepts addressed in this module and therewith to ground their life-long design learning process.

%%%%%%%%%%%%%%%%%%%%%%%%%%%%%%%%%%%%%%%%%%%%%%%%%%%%%%%%%%%%%%%%%%%%%%%%%%%%%%%%%%%%%%%%%%%%%%%%%%%%
%%%%%%%%%%%%%%%%%%%%%%%%%%%%%%%%%%%%%%%%%%%%%%%%%%%%%%%%%%%%%%%%%%%%%%%%%%%%%%%%%%%%%%%%%%%%%%%%%%%%
\section{Fundamental characteristics of design}
\label{sec:whatisdesign}

Without taking any risk of saying something wrong, we can say at a very high level of abstraction that design is \emph{intellectual}, \emph{teleological}, \emph{universal} and \emph{tricky}---four fundamental characteristics which are detailed hereafter. Any further attempt to be more precise about what design is requires making adventurous assumptions or to adopt specific points of view. The next sections provide more or less compatible explanations of what happens in design, shedding different lights on what design is. 

\paragraph{Intellectual.} Design is an intellectual activity in the sense it is virtual, it is a ``pen and paper''-thing {\color{blue}[\stepcounter{slide}Slide \arabic{slide}]}. It is about ``devising a description (plan) for an artefact'' \cite{hybsEvolutionaryProcessModel1992}. It is about creating normative information in the form of definitions, instructions, projections, representations (e.g. CAD models and drawings, assembly instructions, nomenclatures). This normative information is ultimately meant to be executed by physical processes to affect the real world. The distinction between physical and virtual is what distinguishes design and production. Nonetheless, while industrial production is a physical process of execution, it also embeds a part of design, since the normative information it gets delivered is rarely precise enough to be executed \emph{as is} and need to be refined by further detailed design. Also, design may require physical processes in the form of prototyping and testing in order to \emph{validate} information.

\paragraph{Teleological.} Design is teleological in the sense it is about \emph{purposefully} ``devis[ing] courses of action [in order to change] existing situations into preferred ones'' \cite[p. 111]{simon1996sciences} {\color{blue}[\stepcounter{slide}Slide \arabic{slide}]}. Design is what makes the difference between natural sciences and professions (including engineering): while ``[natural] sciences are concerned with how things are [..., a profession] is concerned with how things ought to be, with devising artifacts to attain goals'' \cite[p. 114]{simon1996sciences}. In that sense, design is about defining a way to change a part of the reality to reach a given purpose. In engineering and industry, design is about defining artifacts, that is, artificial things which wouldn't exist without our intervention, such as products or processes. 

\paragraph{Universal.} So defined, design appears to be universal or ubiquitous in the sense it is not only the remit of designers but of all professions {\color{blue}[\stepcounter{slide}Slide \arabic{slide}]}. ``The intellectual activity that produces material artifacts is not different fundamentally from the one that prescribes remedies for a sick patient of the one that devises a new sales plan for a company or a social welfare policy state'' \cite[p. 111]{simon1996sciences}. In that sense, design is common to engineering, architecture, business, medicine, and even arts like painting and music. There is an architectural design as well as there is an interior design, a ``floral design, hair design and funeral design'' \cite{heskett2001past}. This universality makes it difficult to distinguish design from other types of problem solving \cite{hatchuel2003new} and from other disciplines {\color{blue}[\stepcounter{slide}Slide \arabic{slide}]}. ``Everyone can---and does---design. We all design when we plan for something new to happen'' \cite{cross2011design}. Design problems are like ``most of the problems that people face most of the time in everyday life'' \cite{archerDesignDiscipline1979}. This makes design difficult to grasp: do I \emph{design} when I define which route I will take to go and grab a coffee at the cafeteria? Bearing this in mind, industrial design as a ``particular set of skills or organisation appropriate to modern history'' can be seen as a specific case of the unique ``human capacity to shape and make the objects, communications, and systems that serve utilitarian needs and give symbolic meaning to life'' \cite{heskett2001past}, something which ``is inherent within human cognition [and] key part of what makes us human'' \cite{cross2011design}.

\paragraph{Tricky.} What distinguishes design from optimization is that there is no objective function in design. Design is about solving so-called \emph{`wicked problems'}* {\color{blue}[\stepcounter{slide}Slide \arabic{slide}]} \cite{rittelDilemmasGeneralTheory1973}\textsuperscript{\color{Magenta}[key reading]}. Wicked problems cannot be definitively described because of the partly conflicting and ambiguous views of the involved stakeholders. In the absence of objective function, solutions to wicked problems are neither true, false or optimal. They can be at best fairly good, sufficient or satisfying. Consequently, wicked problems do not deliver a way to state objectively when the job is done and the design process can stop. ``Even worse, there is no `solution' in the sense of definitive and objective answer'' (\emph{ibid.}). This particular characteristic of design has tremendous consequences on the organisation of the design activity which will be discussed in the rest of this lecture. A first consequence is that, because it is not possible to enumerate exhaustively all solutions to a wicked problem, solution finding unavoidably requires creativity and experience. It involves `decision making' in the sense of arbitrarily cutting out unmanageable spaces of possible directions to go (see section \ref{sec:GeneralProblemSolving}). A second consequence is that the design activity cannot rely on a solid set of requirement given at the beginning of the process but rather requires to challenge and refine the requirements constantly (see section \ref{sec:illdefined}). 

\section{Design as problem solving}
\label{sec:GeneralProblemSolving}
%%%%%%%%%%%%%%%%%%%%%%%%%%%%%%%%%%%%%%%%%%%%%%%%%%%%%%%%%%%%%%%%%%%%%%%%%%%%%%%%%%%%%%%%%%%%%%%%%%%%
%%%%%%%%%%%%%%%%%%%%%%%%%%%%%%%%%%%%%%%%%%%%%%%%%%%%%%%%%%%%%%%%%%%%%%%%%%%%%%%%%%%%%%%%%%%%%%%%%%%%
Design can be seen as a problem solving procedure. That is, it amounts to recursively build a decision tree starting from an initial situation and exploring possible sequences of future moves to find the most desirable one. In such a tree, each node is a move towards a solution and each tree level is a further level of specificity in defining the solutions {\color{blue}[\stepcounter{slide}Slide \arabic{slide}]}. 

In design like in optimization, a lot of real-life problems are beyond the reach of exact algorithms, the simplest of which being \emph{\href{https://en.wikipedia.org/wiki/Brute-force_search}{brute force search}} {\color{blue}[\stepcounter{slide}Slide \arabic{slide}]}. For these problems, it is neither possible to list all the solutions nor to prove the absolute superiority of one of them. For example, while it is possible to define exact algorithms predicting the best move to do next in chess or to solve the \href{https://en.wikipedia.org/wiki/Travelling_salesman_problem}{travelling salesman problem} {\color{blue}[\stepcounter{slide}Slide \arabic{slide}]}, the volume of operations required to terminate these algorithms is beyond the reach of computational power. In other words: `Only in trivial cases is the computation of the optimum alternative an easy matter. [In] the real world we usually do not have a choice between satisfactory and optimal solutions, for we have only rarely a method of finding the optimum '' \cite[p. 118-120]{simon1996sciences}. Those non-trivial problems are only in the reach of \href{https://en.wikipedia.org/wiki/Heuristic_(computer_science)}{heuristic algorithms}. \href{https://en.wikipedia.org/wiki/Heuristic}{Heuristics} are ``tricks'' gained from experience, strategies which have shown to be advantageous in problem solving. They only guarantee to find fairly good solutions, in contrast to exact algorithms which guarantee to find \emph{the} optimal solution. 

\subsection{Design by analogy}
\label{sec:cbr}
The `wicked' problems of design are of this kind of problems which cannot be solved using brute force. Consequently, in order to identify of possible ``next moves'' towards a solution, designers also need to use heuristics. The corpus of heuristics a designer can use can be represented as a database of connections between classes of problems and classes of solutions they encountered in practice. The ability involved in building these classes out of concrete cases and in retrieving them in the memory is the ability to draw analogies. Analogy is a mechanism we instinctively use in our efforts to understand the world around us or to explain how we understand it. For example, explaining the working principles of electricity often involves a comparison with water flow through a system of pipes {\color{blue}[\stepcounter{slide}Slide \arabic{slide}]}. Analogy also plays an important role in the ability to store information efficiently {\color{blue}[\stepcounter{slide}Slide \arabic{slide}]}. ``\emph{Analogical reasoning}* is based on the idea that problems or experiences outside the one we are currently dealing with may provide some insight or assistance. [...] Analogy is a way of recognizing something that has not been encountered before by associating it with something that has. [It] can be used in common situations, where the previous experience is directly applicable, or in unique or creative situations, where the previous situation shares something with the new situation, but the differences are just as interesting as the similarities'' \cite[p. 1]{maher2014case}. 

The way analogical reasoning works in design is that the designer ``is reminded of the previously solved problem because it has some relevance to the new problem. After the person recalls a previously solved problem, certain aspects of the previous problem's solution are used in the new context and others are not {\color{blue}[\stepcounter{slide}Slide \arabic{slide}]}. For example, [While designing a 18-meter long pedestrian bridge over a busy street, the previous example of] a pedestrian bridge with a span of 15 meters may be recalled [...]. The same design for the superstructure, such as the steel arch, may be used, but the span and sizes of the steel members will change.'' \cite[p. 1-4]{maher2014case}. Analogical reasoning is a critical ability to find quickly relevant ``next moves'' to add to the decision tree.

\subsection{Practical implications}
\label{sec:Practicalimplicationsone}
Understanding design as analogical reasoning implies that the more cases a designer has memorized, the more they are likely to find design alternatives to feed the decision tree. Generating ``a design for a bridge requires not only an understanding of the analysis of bridges, but exposure to examples of several bridge designs.'' \cite[p. 1]{maher2014case}.
 
Ideally, cases are memorized through practice, but can alternatively be gained through learning. For example, designers commonly seek for external stimuli even when they are not actively solving a problem \cite{goncalvesInspirationChoicesThat2016}, for example by browsing design magazines {\color{blue}[\stepcounter{slide}Slide \arabic{slide}]}. Specific design knowledge can also be passed on from designer to designer, either verbally or through textual formulation of design heuristics. Some researchers in design provided sets of design heuristics collected from designers' practice. Some of them are applicable to any design problem, like the \href{https://www.design.iastate.edu/news/2016/08/design-heuristics/}{Iowa State University's 77 heuristics} {\color{blue}[\stepcounter{slide}Slide \arabic{slide}]} or the \href{https://en.wikipedia.org/wiki/40_principles_of_invention}{40 principles of invention} involved in TRIZ {\color{blue}[\stepcounter{slide}Slide \arabic{slide}]}. Some other are targeted at specific design problems like flexible product design {\color{blue}[\stepcounter{slide}Slide \arabic{slide}]} or eco-design {\color{blue}[\stepcounter{slide}Slide \arabic{slide}]}. The use of these heuristics in controlled as well as real design settings has been shown to increase the variety of idea generated and hence to support innovation \cite{yilmazCanExperiencedDesigners2013}. Exposure to a variety of heuristics also proved to help developing expertise in design \cite{yilmazHowDesignersGenerate2015}. 

% end of section - time for take aways and exercise
{\color{PineGreen}
\setlength{\parskip}{1em}
{\emph{Take-aways of this section:
\setlength{\parskip}{0em}
\begin{itemize}
  \item Most design problems cannot be computed to find an optimal solution.
	\item Solution search leans on designer's experience.
	\item Designer's experience grows with exposure to either personally experienced or observed examples.
	\item Designers instinctively infer out of examples design heuristics which guide their action.
	\item To increase experience: be curious, read design magazines, analyze existing designs, tinker around.
\end{itemize}
}}
}

\section{Design as exploration}
\label{sec:illdefined}
%%%%%%%%%%%%%%%%%%%%%%%%%%%%%%%%%%%%%%%%%%%%%%%%%%%%%%%%%%%%%%%%%%%%%%%%%%%%%%%%%%%%%%%%%%%%%%%%%%%%
%%%%%%%%%%%%%%%%%%%%%%%%%%%%%%%%%%%%%%%%%%%%%%%%%%%%%%%%%%%%%%%%%%%%%%%%%%%%%%%%%%%%%%%%%%%%%%%%%%%%
Solving the traveling salesman problem requires using heuristics because it is \emph{impractical} to compute all possible solutions. Solving design problems is not only impractical, but \emph{impossible}. Because designs problems are `wicked', they are also ill-defined, like equation systems where there are more unknowns than equations. To say it short, design amounts to ``problem solving without a goal'' \cite[p. 106]{simon1996sciences} {\color{blue}[\stepcounter{slide}Slide \arabic{slide}]}. Ill-defined problems are those that have less specific criteria for knowing when the problem is solved, and do not supply all the information required for solution. The distinction between ill- and well-defined problems is based on the amount of information guiding the search of solution given in the task environment. 

This is greatly formulated by Archer \cite{archerDesignDiscipline1979}: ``An ill-defined problem is one in which the requirements, as given, do not contain sufficient information to enable the designer to arrive at means of meeting those requirements simply by transforming, reducing, optimizing or superimposing the given information alone. Some of the necessary further information may be discoverable simply by searching for it, some may be generateable by experiment, some may turn out to be statistically variable, some may be vague or unreliable, some may arise from capricious fortune or transitory preference and some may be actually unknowable.''

That ``some information may arise from capricious fortune'' means that ill-defined problems are subject to `discoveries' potentially challenging the formulation of the problem: To ``discover gold, one does not even have to be looking for it (although frequently one is), and if silver or copper shows up instead of gold, that outcome will usually be welcome too. The test that something has been discovered is that something new has emerged that could not have been predicted with certainty and that the new thing has value or interest of some kind.'' This aspect makes of design a rather unpredictable and opportunistic activity requiring designers to be able to deal with uncertainty.

\subsection{Co-evolution of problem and solution spaces}
\label{sec:coevolution}
In this sense, design is more than problem solving, more than the simple search of a satisfactory solution to a given problem. It is also about \emph{refining a problem}. In design, ``the `problem' is [...] not the statement of requirements. [... It] is \emph{obscurity} about the requirements, the practicability of envisageable provisions and/or misfit between the requirements and the provisions'' \cite[emphasis is not in the original text]{archerDesignDiscipline1979}. Design is about reducing uncertainty about ``what designers need to know about the problem'' and which ``only becomes apparent as [they] are trying to solve it'' \cite{cross2011design}. 

All this means that the design activity involves a co-evolution of the problem and the solution \cite{maherFormalisingDesignExploration1996} {\color{blue}[\stepcounter{slide}Slide \arabic{slide}]}. ``[The] formulation of the problem at any stage is not final [...]. As the design progresses, the designer learns more about possible problem and solution structures as new aspects of the situation become apparent and the inconsistencies inherent in the formulation of the problem are revealed. As a result, [...] the problem and the solution are redefined'' (\emph{ibid.}, p. 5). Consequently, it appears that design involves an ``iterative interplay to `fix' a problem from the problem space and to `search' plausible solutions from the corresponding solution space'' (\emph{ibid.}, p. 1). Externalising the design (e.g. sketching it) allows a dialogue between the problem and solution to evolve, it allows questions to be raised about the width of the solution space and the range of acceptable requirements, what in turns allows decisions to be made to reduce those spaces and to converge towards a fixed design. 

In summary, design is about refining \emph{both} a problem \emph{and} a compatible solution to this problem {\color{blue}[\stepcounter{slide}Slide \arabic{slide}]}. Consequently, the design activity does not start with precise product specifications but produces them out of rough ``hopes and aspirations'', eventually gained from a study of customer needs \cite[p. 73]{ulrichProductDesignDevelopment2011}. The exit condition of this interactive process does not only lie in the consistency between the problem and the solution (when the product design is said to ``meet'' the specifications or ``fulfill'' the requirements), but in external factors: for example, time and money went out, or the team is satisfied with the degree of precision achieved in the formulation of both problems and solutions. 

Attempting to define the best moment in the product development process for defining product requirements, Ulrich and Eppinger's reference lecture on product design and development says: ``In an ideal world, the team would establish the product specifications \emph{once} early in the development process and \emph{then} proceed to design and engineer the product to \emph{exactly} meet those specifications.'' \cite[p. 73, emphases are not in the original text]{ulrichProductDesignDevelopment2011}. This sentence begins with ``in an ideal world'' because it acknowledges that problems do not precede solutions in design, although it would be way easier to handle product development as if it would be the case. 

\subsection{Practical implications}
\label{sec:practicalimplicationstwo}

The co-evolution of the problem and solution space is illustrated by the difficulty to implement strictly sequential process models such as the \href{https://en.wikipedia.org/wiki/Waterfall\_model}{Waterfall model of engineering design} {\color{blue}[\stepcounter{slide}Slide \arabic{slide}]}, the Pahl \& Beitz design process \cite{pahlEngineeringDesignSystematic2007} and the VDI 2221 \cite{vdi1993design}. These product development processes start with defining the requirements. While this may work in cases where design has low chances to lead to unpredicted outcomes (e.g. slight redesign of a product based on a well-managed technology), it may not be beneficial in radically new product design. Indeed, in these cases, the refined understanding of the problem yielded by the design activity may lead to challenging the requirements, an event which requires iteration and which does not fit with the logic of stage-gate processes. This contradiction is solved in the VDI 2221 by accompanying the sequential process by an ongoing requirement refinement task \cite{vdi1993design} {\color{blue}[\stepcounter{slide}Slide \arabic{slide}]}.

In reaction to these difficulties, the software branch came out in the 90's with new project management principles under the umbrella of ``agile software development'' \cite{beck2001manifesto} which then spread to other engineering disciplines like engineering design. Among the key principles of agile are a ``close collaboration between the development team and business stakeholders'' in order to make the users participate in the incremental refinement of requirements. This requires delivering ``working software frequently, from a couple of weeks to a couple of months, with a preference to the shorter timescale'', discussing the results with users in ``face-to-face conversation'' and being open for ``changing requirements, even late in development''. Examples of agile management methods in software engineering are \href{https://en.wikipedia.org/wiki/Extreme_programming}{Extreme Programming (XP)} {\color{blue}[\stepcounter{slide}Slide \arabic{slide}]} which is based on short development cycles and \href{https://en.wikipedia.org/wiki/Scrum_(software_development)}{Scrum} which is based on weekly design sprints.

Considering design as a co-evolution of both problem and solution spaces also helps understand why customers may have difficulties to express their requirements in practice. A famous (\href{https://quoteinvestigator.com/2011/07/28/ford-faster-horse/}{but apocryphal}) quote attributed to Henry Ford says: ``If I had asked people what they wanted, they would have said faster horses'' {\color{blue}[\stepcounter{slide}Slide \arabic{slide}]}. This iconic case illustrates that requirements may emerge only after a solution has been proposed. This problem is also a part of communication problem the widely encountered in software engineering and humorously referred to as the `\href{https://en.wikipedia.org/wiki/Tree_swing_cartoon}{Tree Swing Cartoon}' {\color{blue}[\stepcounter{slide}Slide \arabic{slide}]}. End users may be delivered with a solution which does not match their need even if they have been asked what they wanted at the beginning of the process. If they do not take part in the whole product development process, they don't participate in the iterative refinement of the problem and the solution. They don't profit from the better understanding gained by the development team, which may converge towards a direction which is not useful for the users. Avoiding this pitfall is another benefit of agile project management methods. 

% end of section - time for take aways and exercise
{\color{PineGreen}
\setlength{\parskip}{1em}
{\emph{Take-aways of this section:
\setlength{\parskip}{0em}
\begin{itemize}
  \item The word of saying ``clearly stating the problem is half the way to solve it'' also apply to design. 
  \item In most cases it is not realistic to expect specifications can be all set at the beginning of the design process.
	\item Do not accept the brief `as given', your role as designer is to question it.
	\item Stage-gate processes are relevant for design projects with low probability of making discoveries.
	\item Agile processes are relevant for design projects with low probability of making discoveries.
	\item The process of refining the requirements may need to involve the users so the designers don't converge towards an unwanted solution.
\end{itemize}
}}
}

\section{Design as sense-making}
\label{sec:sensemaking}
%%%%%%%%%%%%%%%%%%%%%%%%%%%%%%%%%%%%%%%%%%%%%%%%%%%%%%%%%%%%%%%%%%%%%%%%%%%%%%%%%%%%%%%%%%%%%%%%%%%%
%%%%%%%%%%%%%%%%%%%%%%%%%%%%%%%%%%%%%%%%%%%%%%%%%%%%%%%%%%%%%%%%%%%%%%%%%%%%%%%%%%%%%%%%%%%%%%%%%%%%
if there is neither a problem nor a solution in deisgn, design cannot be defined as problem solving. So what is it?
reduce incertainties, learning
schoen reflective practice
https://www.youtube.com/watch?v=wMLSrqYk0UE
https://www.jstor.org/stable/pdf/1511512.pdf?casa_token=0dLVK-gaMowAAAAA:9mVfSpdLdk9y8rnSrmAit9bM2TPjCA-AkRafnVlyvDNWwZ7JS2Enz5fw4VIxa2cw7NhrnOQM-qZpTHwWR6vUXlEI0bG0qqsa_zW_m6FbQmhGvBsVKBs
https://www.youtube.com/watch?v=wMLSrqYk0UE

%%%%%%%%%%%%%%%%%%%%%%%%%%%%%%%%%%%%%%%%%%%%%%%%%%%%%%%%%%%%%%%%%%%%%%%%%%%%%%%%%%%%%%%%%%%%%%%%%%%%
%%%%%%%%%%%%%%%%%%%%%%%%%%%%%%%%%%%%%%%%%%%%%%%%%%%%%%%%%%%%%%%%%%%%%%%%%%%%%%%%%%%%%%%%%%%%%%%%%%%%
\section{C-K theory}
\label{sec:CK}
The ``C-K theory'' introduced by Hatchuel and Weil \cite{hatchuel2003new}\footnote{All quoted sentences in this section are taken from this source (\cite{hatchuel2003new})} defines design as the parallel expansion of two ensembles: the knowledge and concept space {\color{blue}[\stepcounter{slide}Slide \arabic{slide}]}. 

The ``knowledge space'' K is defined as the ``space of propositions that have a logical status'' for a given designer or team of designers. In other words, K is the set of all propositions a designer or a group of designers know to be true or false. The elements of this space are propositions designers can rely on and don't have to explore because they know these to be true or false. For example, a designer may know that knives are made to cut and airplanes can fly, they can take these propositions for granted. 

The ``concept space'' C is defined as the space of propositions that have no logical status. In other words, C is the set of propositions a designer or the group of designers do not know whether they are true or false. For example, a designer may never have experienced a knife that can fly or a plane that can cut. The propositions ``a knife that can fly'' and ``a plane that can cut'' are neither true or false for them as long as they don't experience these or get proven it is not possible to do such things. Propositions of C are hypothetical ideas thrown in a design process waiting to be validated and transformed into knowledge. These are the creative bits of design. 

From this, design is defined as a sequential process of cross-fertilization between C and K spaces through four types of operations building the so-called ``design square'' {\color{blue}[\stepcounter{slide}Slide \arabic{slide}]}:
\begin{itemize}
	\item $K\rightarrow C$: this operator creates elements in C (new concepts) from elements of K (bits of knowledge). A concept can be generated by combining elements of knowledge. A designer knowing the proposition ``airplanes can fly'' and ``there is a planet called Venus'' could come to the proposition of an ``airplaine flying on Venus''. This proposition remains a concept (that is, an element of C) as long as he has not been proven whether it is possible to make an airplane fly on Venus or not (the answer \href{https://whatif.xkcd.com/book/}{here}).
	\item $C\rightarrow K$: this operator creates elements in K (bits of knowledge) from elements of C (concepts). Transforming concepts into knowledge corresponds to what is generally called `concept validation'. This can be done in various ways, such as ``consulting an expert, doing a test, an experimental plan, a prototype, a mock-up''. An element of C is transformed in an element of K when it is given a logical status (true or false).
	\item $C\rightarrow C$: this operator creates concepts out of concepts. It amounts to jump onto an idea to create a new one.
	\item $K\rightarrow K$: this operator creates knowledge out of knowledge. This brings us back to the conventional rules of deductive reasoning used for proving mathematical theorems or in syllogistic reasoning such as: ``All men are mortal / Socrates is a man / Therefore, Socrates is mortal.'' 
\end{itemize}

\subsection{Application example}
\label{sec:CKExample}
The case of a ``nail holder avoiding to hurt one's hand while hammering'' reproduced here is reported in \cite{hatchuel2004ck}.
The reasoning process starts with the concept \{safe knocking a nail\} {\color{blue}[\stepcounter{slide}Slide \arabic{slide}]}, which is a combination of the elements of knowledge \{safe\} and \{knocking a nail\}. The designer or the design team know about \{Knocking a nail\} and can rephrase it as \{hammer in right hand, nail in left hand, energy given by shocks\}. Having recalled the element of K \{hammer\}, the designers can choose to challenge it and explore two refined concept alternatives \{Safe knocking a nail with a hammer\} and \{Safe knocking a nail without a hammer\}. As well, having recalled the element of K \{nail in left hand\}, they can challenge it and explor two further refined concept alternatives \{Safe knocking a nail with a hammer and with the nail in the left hand\} and \{Safe knocking a nail with a hammer and without holding the nail in the left hand\}. The designer team also knows about \{safe\} meaning \{without knocking on fingers\}. From this, they can derive alternative concepts to hold the nail without having the risks of knocking it with the hammer, such as \{avoid the hammer to derive from the desired trajectory\} or \{protect the hand from the hammer\}. These concepts can be refined into more precise concepts \{a trajectory control device\} and a \{hilt\} or \{protecting glove\}. 

The same tree-shaped recursive search algorithm can be continued from any element from C or K, such as \{safe hammering with hammer in right hand and left hand not holding the nail\} {\color{blue}[\stepcounter{slide}Slide \arabic{slide}]}. The process stops when the leaves of a given branch of interest in the exploration tree are validated, that is, they are fed back in K using the $C\rightarrow K$ operator.

\subsection{Practical implications}
\label{sec:practicalimplicationsthree}
The C-K theory can be used to log the outcomes of a creativity session. Therewith, it offers a way to structure further creative processes and to systematically explore new alternatives by going back where a branch may have been under-explored. If further provides research with a formalism to record and analyze design processes. 

Beyond this, the C-K theory underlines the important role played by prototyping in design and explains why companies have integrated R\&D departments. The design process creates concepts that need to be validated and fed back into the knowledge space so they can be put into practice. While the validation of some concepts can be done with quick prototyping or mock-up techniques, the validation of some concepts may takes years of efforts. These more ambitious and visionary concepts requiring technological development are the job of the `R' part in R\&D. Engineers working in the `R' of `R\&D' perform research in the sense that they intend to expand the knowledge space in large design endeavors. This research is `applied' in the sense it remains teleological, that is, it is part of the design process (of the `D' of `R\&D') its finality is ``bring a concept to some form of ``reality''''.

The C-K theory also helps avoiding the difficulty to differentiate problems and solutions due to the `wicked' nature of design problems. It avoids the necessity to postulate that the aim of the design process is to make a product definition `meeting' product specification---and we have seen in section \ref{sec:coevolution} that we should not postulate that specifications precede design. Instead, the C-K theory does not make the difference between specifications and product definition which are both considered as `concepts' as long as they haven't been validated.

{\color{red}
C-K Design Theory and Design Principle Formulation and Reuse (Kruse and Seidel) %https://cora.ucc.ie/bitstream/handle/10468/4456/3503.pdf?sequence=1&isAllowed=y
REmco van der Lugt - Sketching
Tom Howard - use of stimuli
Chris Snider - 2 different ways of doing detailed design
}
% end of section - time for take aways and exercise
{\color{PineGreen}
\setlength{\parskip}{1em}
{\emph{Take-aways of this section:
\setlength{\parskip}{0em}
\begin{itemize}
  \item Design is a process of expansion of the designer's knowledge and available concepts.
  \item Design terminates when concepts of interest are turned into kwowledge, that is, their feasibility has been proven and they can be turned into reality.
	\item Concepts are created by recombination of elements of knowledge.
	\item Research (in the sense of technological development) is a part of design.
\end{itemize}
}}
}

\section*{Credits}
\label{sec:credits}
%%%%%%%%%%%%%%%%%%%%%%%%%%%%%%%%%%%%%%%%%%%%%%%%%%%%%%%%%%%%%%%%%%%%%%%%%%%%%%%%%%%%%%%%%%%%%%%%%%%%
%%%%%%%%%%%%%%%%%%%%%%%%%%%%%%%%%%%%%%%%%%%%%%%%%%%%%%%%%%%%%%%%%%%%%%%%%%%%%%%%%%%%%%%%%%%%%%%%%%%%
These works are released under a \href{https://creativecommons.org/licenses/by/4.0/}{Creative Commons Attribution 4.0 International License}.


\bibliographystyle{ieeetr}
\bibliography{../References}
\end{document}
