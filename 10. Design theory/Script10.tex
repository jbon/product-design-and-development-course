\documentclass{article}

% style elements
% Slide numbers {\color{blue}[\stepcounter{slide}Slide \arabic{slide}]}
% Key readings \textsuperscript{\color{Magenta}[key reading]}

\usepackage[utf8]{inputenc}
\usepackage{titling}
\usepackage{hyperref}
\usepackage{color}
\usepackage[dvipsnames]{xcolor}
\usepackage{comment}
\usepackage{textcomp}

\newcounter{slide}
\newcounter{exercise}

\newcommand{\subtitle}[1]{%
  \posttitle{%
    \par\end{center}
    \begin{center}\large#1\end{center}
    \vskip0.5em}%
}

\begin{document}


\title{Design theory}

\author{Product Design and Development (ME30294). Lecture 10. \\ Jérémy Bonvoisin, Dept. Mech. Eng., University of Bath}
\date{Last update: \today}

\maketitle

\begin{abstract}
This lecture introduces some descriptive models of the design activity in order to explain the relevance of the more practical content delivered in this module. 
Covered topics
- Linkography 
- Gero
- CK
- design as decision making / finding the good compromise between criteria / balancing trade offs
\end{abstract}

\tableofcontents

\section{Design as a temporal process}
\label{sec:temporal}
%%%%%%%%%%%%%%%%%%%%%%%%%%%%%%%%%%%%%%%%%%%%%%%%%%%%%%%%%%%%%%%%%%%%%%%%%%%%%%%%%%%%%%%%%%%%%%%%%%%%
%%%%%%%%%%%%%%%%%%%%%%%%%%%%%%%%%%%%%%%%%%%%%%%%%%%%%%%%%%%%%%%%%%%%%%%%%%%%%%%%%%%%%%%%%%%%%%%%%%%%
\subsection{Co-evolution of problem and solution spaces}
\label{sec:coevolution}
Attempting to define the best moment in the product development process for defining product requirements, Ulrich and Eppinger's reference lecture on product design and development says: ``In an ideal world, the team would establish the product specifications \emph{once} early in the development process and \emph{then} proceed to design and engineer the product to \emph{exactly} meet those specifications.'' \cite[p. 73, emphases are not in the original text]{ulrichProductDesignDevelopment2011}. This sentence begins with ``in an ideal world'' because it acknowledges that problems do not precede solutions in design, although it would be way easier to handle product development as if it would be the case. Instead, the design activity involves a co-evolution of the problem and the solution \cite{maherFormalisingDesignExploration1996} {\color{blue}[\stepcounter{slide}Slide \arabic{slide}]}. ``[The] formulation of the problem at any stage is not final [...]. As the design progresses, the designer learns more about possible problem and solution structures as new aspects of the situation become apparent and the inconsistencies inherent in the formulation of the problem are revealed. As a result, [...] the problem and the solution are redefined'' (\emph{ibid.}, p. 5). Consequently, it appears that design involves an ``iterative interplay to `fix' a problem from the problem space and to `search' plausible solutions from the corresponding solution space'' (\emph{ibid.}, p. 1).

In summary, design is about refining \emph{both} a problem \emph{and} a compatible solution to this problem. Consequently, the design activity does not start with precise product specifications but produces them out of rough ``hopes and aspirations'', eventually gained from a study of customer needs \cite[p. 73]{ulrichProductDesignDevelopment2011} {\color{blue}[\stepcounter{slide}Slide \arabic{slide}]}. The exit condition of this interactive process does not only lie in the consistency between the problem and the solution, but in external factors: for example, time and money went out, or the team is satisfied with the degree of precision achieved in the formulation of both problems and solutions. 

\subsection{Design as an ill-defined problem}
\label{sec:DesignAsAnIllDefinedProblem}
This situation is due to the very nature of design activity which amounts to ``problem solving without a goal'' \cite[p. 106]{simon1996sciences}. Problems in design are ill-defined, like equation systems where there are more unknowns than equations. Ill-defined problems are those that have less specific criteria for knowing when the problem is solved, and do not supply all the information required for solution. The distinction between ill- and well-defined problems is based on the amount of information guiding the search of solution given in the task environment. Ill-defined problems are moreover subject to `discoveries', in the sense that to ``discover gold, one does not even have to be looking for it (although frequently one is), and if silver or copper shows up instead of gold, that outcome will usually be welcome too. The test that something has been discovered is that something new has emerged that could not have been predicted with certainty and that the new thing has value or interest of some kind.'' This aspect makes of design a rather unpredictable and opportunistic activity.

\subsection{Practical implications}
\label{sec:practicalimplications}

The co-evolution of the problem and solution space is illustrated by the difficulty to implement in practice rigid stage-gate process models such as the \href{https://en.wikipedia.org/wiki/Waterfall\_model}{Waterfall model of engineering design} {\color{blue}[\stepcounter{slide}Slide \arabic{slide}]} or the Pahl \& Beitz design process \cite{pahlEngineeringDesignSystematic2007}. These sequential design approaches start with defining the requirements. The design then only starts once the requirements are defined. While this process may work in cases where design has low chances to lead to unpredicted outcomes (e.g. slight redesign of a product based on a well-managed technology), it may not be beneficial in radically new product design. Indeed, in these cases, the refined understanding of the problem yielded by the design activity may lead to challenging the requirements, an event which requires iteration and which does not fit with the logic of stage-gate processes.

In reaction to these difficulties, the software branch came out in the 90's with new project management principles under the umbrella of ``agile software development'' \cite{beck2001manifesto} which then spread to other engineering disciplines like engineering design. Among the key principles of agile are a ``close collaboration between the development team and business stakeholders'' in order to make the users participate in the incremental refinement of requirements. This requires delivering ``working software frequently, from a couple of weeks to a couple of months, with a preference to the shorter timescale'', discussing the results with users in ``face-to-face conversation'' and being open for ``changing requirements, even late in development''. Examples of agile management methods in software engineering are \href{https://en.wikipedia.org/wiki/Extreme_programming}{Extreme Programming (XP)} {\color{blue}[\stepcounter{slide}Slide \arabic{slide}]} which is based on short development cycles and \href{https://en.wikipedia.org/wiki/Scrum_(software_development)}{Scrum} which is based on weekly design sprints.

Considering design as a co-evolution of both problem and solution spaces also helps understand why customers may have difficulties to express their requirements in practice. A famous (\href{https://quoteinvestigator.com/2011/07/28/ford-faster-horse/}{but apocryphal}) quote attributed to Henry Ford says: ``If I had asked people what they wanted, they would have said faster horses'' {\color{blue}[\stepcounter{slide}Slide \arabic{slide}]}. This iconic case illustrates that requirements may emerge only after a solution has been proposed. This problem is also a part of communication problem the widely encountered in software engineering and humorously referred to as the `\href{https://en.wikipedia.org/wiki/Tree_swing_cartoon}{Tree Swing Cartoon}' {\color{blue}[\stepcounter{slide}Slide \arabic{slide}]}. End users may be delivered with a solution which does not match their need even if they have been asked what they wanted at the beginning of the process. If they do not take part in the whole product development process, they don't participate in the iterative refinement of the problem and the solution. They don't profit from the better understanding gained by the development team, which may converge towards a direction which is not useful for the users. Avoiding this pitfall is another benefit of agile project management methods. 

% end of subsection - time for take aways and exercise
{\color{PineGreen}
\setlength{\parskip}{1em}
{\emph{Take-aways of this section:
\setlength{\parskip}{0em}
\begin{itemize}
  \item The word of saying ``clearly stating the problem is half the way to solve it'' also apply to design. 
  \item In most cases it is not realistic to expect specifications can be all set at the beginning of the design process.
	\item Stage-gate processes are relevant for design projects with low probability of making discoveries.
	\item Agile processes are relevant for design projects with low probability of making discoveries.
	\item The process of refining the requirements may need to involve the users so the designers don't converge towards an unwanted solution.
\end{itemize}
}}
}

\section{Design as a cognitive activity}
\label{sec:cognitive}
%%%%%%%%%%%%%%%%%%%%%%%%%%%%%%%%%%%%%%%%%%%%%%%%%%%%%%%%%%%%%%%%%%%%%%%%%%%%%%%%%%%%%%%%%%%%%%%%%%%%
%%%%%%%%%%%%%%%%%%%%%%%%%%%%%%%%%%%%%%%%%%%%%%%%%%%%%%%%%%%%%%%%%%%%%%%%%%%%%%%%%%%%%%%%%%%%%%%%%%%%
One consequence of the complex and ill-defined nature of design problems is that it is not possible to use a \href{https://en.wikipedia.org/wiki/Brute-force_search}{``brute force''} solving algorithm in design {\color{blue}[\stepcounter{slide}Slide \arabic{slide}]}. That is, it is neither possible to list all solutions to a given problem nor to prove the absolute superiority of a given solution. ``Only in trivial cases is the computation of the optimum alternative an easy matter. [In] the real world we usually do not have a choice between satisfactory and optimal solutions, for we have only rarely a method of finding the optimum '' \cite[p. 118-120]{simon1996sciences}. Instead, designers use \href{https://en.wikipedia.org/wiki/Heuristic}{heuristics} to find fairly good solutions after only moderate search. Their ``algorithm'' is to build a set of thinkable alternative solutions to a problem and to evaluate how promising these solutions are. The most promising solutions are further explored by iterating the algorithm at a greater level of specificity, therewith building a decision tree {\color{blue}[\stepcounter{slide}Slide \arabic{slide}]}. 

\subsection{Design by analogy}
\label{sec:cbr}
The identification of possible solutions is fueled by the designers' experience, which can be represented as a database of connections between the problems and solutions they encountered in practice. Examples a designer has been exposed to are stored in their memory as `cases' to be used in \emph{analogical reasoning}*. ``Analogical reasoning is based on the idea that problems or experiences outside the one we are currently dealing with may provide some insight or assistance. [...] Analogy is a way of recognizing something that has not been encountered before by associating it with something that has. [It] can be used in common situations, where the previous experience is directly applicable, or in unique or creative situations, where the previous situation shares something with the new situation, but the differences are just as interesting as the similarities'' \cite[p. 1]{maher2014case}. More generally, analogy is a mechanism we instinctively use in our efforts to understand the world around us or to explain how we understand it.  For example, explaining the working principles of electricity often involves a comparison with water flow through a system of pipes {\color{blue}[\stepcounter{slide}Slide \arabic{slide}]}. 

The way analogical reasoning works is that the designer ``is reminded of the previously solved problem because it has some relevance to the new problem. After the person recalls a previously solved problem, certain aspects of the previous problem's solution are used in the new context and others are not {\color{blue}[\stepcounter{slide}Slide \arabic{slide}]}. [While designing a 18-meter long pedestrian bridge over a busy street, the previous example of] a pedestrian bridge with a span of 15 meters may be recalled [...]. The same design for the superstructure, such as the steel arch, may be used, but the span and sizes of the steel members will change.'' \cite[p. 1-4]{maher2014case}.

\subsection{Practical implications}
\label{sec:heuristics}
This suggests that the more cases a designer has memorized, the more they are likely to find design alternatives to feed the decision tree. For example, generating``a design for a bridge requires not only an understanding of the analysis of bridges, but exposure to examples of several bridge designs.'' \cite[p. 1]{maher2014case}. This also explains why designers commonly seek for external stimuli even when they are not actively solving a problem \cite{goncalvesInspirationChoicesThat2016}, for example by browsing design magazines {\color{blue}[\stepcounter{slide}Slide \arabic{slide}]}.

Also, experienced designers intuitively infer \emph{design heuristics}* out of their memorized cases. Design heuristics are ``directives [...] which provide design process direction to increase the chance of reaching a satisfactory but not necessarily optimal solution'' \cite{fuDesignPrinciplesLiterature2016} {\color{blue}[\stepcounter{slide}Slide \arabic{slide}]}. These first implicit heuristics can be elicited and collected to be communicated to other designers. Example of heuristics bases are the \href{https://www.design.iastate.edu/news/2016/08/design-heuristics/}{Iowa State University's 77 heuristics} {\color{blue}[\stepcounter{slide}Slide \arabic{slide}]} or the \href{https://en.wikipedia.org/wiki/40_principles_of_invention}{40 principles of invention} involved in TRIZ {\color{blue}[\stepcounter{slide}Slide \arabic{slide}]}. The use of these heuristics in controlled as well as real design settings has been shown to increase the variety of idea generated and hence to support innovation \cite{yilmazCanExperiencedDesigners2013}. Exposure to a variety of heuristics also proved to help developing expertise in design \cite{yilmazHowDesignersGenerate2015}. 

% end of subsection - time for take aways and exercise
{\color{PineGreen}
\setlength{\parskip}{1em}
{\emph{Take-aways of this section:
\setlength{\parskip}{0em}
\begin{itemize}
  \item Most design problems cannot be computed to find an optimal solution.
	\item Solution search leans on designer's experience.
	\item Designer's experience grows with exposure to either personally experienced or observed examples.
	\item Designers instinctively infer out of examples design heuristics which guide their action.
	\item To increase experience: be curious, read design magazines, analyze existing designs, tinker around.
\end{itemize}
}}
}

\bibliographystyle{ieeetr}
\bibliography{../References}
\end{document}
