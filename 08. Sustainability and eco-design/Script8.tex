\documentclass{article}

% style elements
% Slide numbers {\color{blue}[\stepcounter{slide}Slide \arabic{slide}]}
% Key readings \textsuperscript{\color{Magenta}[key reading]}

\usepackage[utf8]{inputenc}
\usepackage{titling}
\usepackage{hyperref}
\usepackage{color}
\usepackage[dvipsnames]{xcolor}
\usepackage{comment}
\usepackage{textcomp}
\usepackage{framed}

\newcounter{slide}
\newcounter{exercise}

\newcommand{\subtitle}[1]{%
  \posttitle{%
    \par\end{center}
    \begin{center}\large#1\end{center}
    \vskip0.5em}%
}

\begin{document}

\title{Sustainability and eco-design}
%\subtitle{Course Product Design and Development. Lecture 8.}

\author{Product Design and Development (ME30294). Lecture 8. \\ Jérémy Bonvoisin, Dept. Mech. Eng., University of Bath}
\date{Last update: \today}

\maketitle

\begin{abstract}
This lecture discusses the implications of PDD in terms of environmental sustainability and gives an overview of the corresponding mitigation measures referred under the umbrella of eco-design---a term defined as the maximisation of the ratio between the product functionality and the associated environmental impacts. In a first section, the lecture recalls the general notions of sustainability and environmental impact and links these general concepts to the design-specific concepts of product life cycle and functional unit. The second section introduces the rationale of eco-design and provides an overview of eco-design strategies. 

{{\it Keywords.} sustainability; environmental impacts; consumption; pollution; resource depletion; product life cycle; functional unit; eco-efficiency; elementary flows; hot spots; environmental assessment; LCA; MECO Matrix; DfE guidelines; eco-ideation mechanisms.}
\end{abstract}

\tableofcontents

\section{Prologue: why this lecture?}
\label{sec:prologue}
A distinctive characteristic of the typical middle-class family---a model the largest part of the fast growing world population longs for---is an overabundance of possessions, stuff and gadgetry \cite{menzelMaterialWorldGlobal1995}. And a rising overabundance of it: the number of energy using products typically found in a household have more than doubled between 1970 and 2000 \cite{owenRiseMachinesReview2006} and may have continued to rise in the meantime {\color{blue}[\stepcounter{slide}Slide \arabic{slide}]}. For certain populations (e.g. US and western countries), this evolution even comes to a point where stuff and unused clutter becomes an overwhelming psychological burden for most middle-class families \cite{arnoldChangingAmericanHome2007}. A point where consumption becomes counter-productive.

The global over-consumption of stuff already `overshoots' of the Earth's limited capacity to deliver enough resources and even to deliver a stable living environment \cite{rockstromPlanetaryBoundariesExploring2009} {\color{blue}[\stepcounter{slide}Slide \arabic{slide}]}. Neither can the Earth offer enough resources to give everyone access to the current middle-class family model, nor can It do it for the next generations issuing from the people who already access this model \footnote{To hear an interesting opinion about the role of consumption in over-exploitation of Earth's resources, see the excellent video \href{https://www.youtube.com/watch?v=9GorqroigqM&vl=fr}{``The Story of Stuff''}. \textsuperscript{\color{Magenta}[key reading]}}. In the ethical debate about resource distribution, the term `sustainability' speaks for the search of a life-style which can be adopted across humanity and perpetuated to next generations. And there is hope sustainability can be reached, because a lot of things we do are just inefficient {\color{blue}[Slides \stepcounter{slide}\arabic{slide}-\stepcounter{slide}\arabic{slide}]}.

This lecture is meant to provide you---as a designer---with an overview of the role and potential of design in contributing to sustainability. 

\section{Products and sustainability}
\label{sec:sustainability}
%%%%%%%%%%%%%%%%%%%%%%%%%%%%%%%%%%%%%%%%%%%%%%%%%%%%%%%%%%%%%%%%%%%%%%%%%%%%%%%%%%%%%%%%%%%%%%%%%%%%
%%%%%%%%%%%%%%%%%%%%%%%%%%%%%%%%%%%%%%%%%%%%%%%%%%%%%%%%%%%%%%%%%%%%%%%%%%%%%%%%%%%%%%%%%%%%%%%%%%%%

According to the most widely accepted definition, \emph{sustainable development} or \emph{sustainability}* ``is development that meets the needs of the present without compromising the ability of future generations to meet their own needs'' \cite{brundtland1987our}. The ``ability of future generation to meet their own needs'' is generally understood as depending from the preservation of three types of capital: the environmental\footnote{in other words: the `nature', the `natural resources', including living biological stock, usable mineral resources, breathable air and drinkable water, beautiful landscapes, etc.}, social\footnote{in other words: human well-being, happiness, and its constituents being eventually, health, education, equality of chances, freedom, etc.}, and economical capitals\footnote{in other words: money and other forms of marketable assets.}. 

Further efforts to break down this programmatic definition into concrete terms are inevitably bound to political implications and are therefore subject to diverging interpretations. One main topic of disagreement is whether capitals are interchangeable, e.g. whether a certain amount of social capital can be sacrificed for a bit of environmental capital. In the one hand, proponents of the so-called \emph{weak sustainability} consider the three capitals as interchangeable and of equal importance and seeks for a preservation or growth of their total balance. This school of thought tends to consider the environment from an utilitarian point of view. It is 1) the supplier of all the resources we need to satisfy human needs and 2) the repository of all we don't need anymore. Following this idea, the function of natural cycles is to absorb wastes and to turn them into restored resources. Caring about the environment is a matter of ensuring the further fulfillment of human basic needs (e.g. breathe, eat, drink, reproduce) and well-being (to which may contribute things like mediated communication, mechanised transportation, cultural creation, entertainment, comfort, representation). On the other hand, proponents of the so-called \emph{strong sustainability} consider there should be no substitution. This school of thought tends to consider the environment from an idealistic perspective and to speak for nature rights. It assumes that preservation of the social capital is the objective of human activity; those of the economical capital is a mean to achieve it; those of the environmental capital is understood as a condition. Weak sustainability is generally represented as a Venn diagram and strong sustainability as concentric circles {\color{blue}[\stepcounter{slide}Slide \arabic{slide}]}. Adhesion to one or the other school of thought is ultimately a question of \emph{Weltanschauung}, one's world view, their own philosophy.

Without taking position in this debate, we can state that the sustainability of human activities relates to the extent to which they:
\begin{itemize}
	\item create social value (they are \emph{useful} to someone). Social value is assumed to be positive and is to be maximized. 
	\item contributes to environmental degradation (they have \emph{environmental impacts*}). Environmental degradation is assumed to be negative and is to be minimized.
	\item participate to the creation of economical value added (they can be charged for money). From the point of view of the firm, economical value added is considered as positive and is to be maximized to ensure viability. The weak and strong sustainability disagree on whether economical value added has to be maximized from a macroeconomic point of view.
\end{itemize}

Consequently, in the context of this lecture, we consider that caring for sustainability consists in maximising social value and minimising the associated environmental impacts while contributing to economical viability. In the following, we further consider that the economical and social aspects of sustainability are self-explaining and therefore focus on the environmental dimension.

\subsection{Environmental impacts}
\label{sec:EnvironmentalImpacts}
%%%%%%%%%%%%%%%%%%%%%%%%%%%%%%%%%%%%%%%%%%%%%%%%%%%%%%%%%%%%%%%%%%%%%%%%%%%%%%%%%%%%%%%%%%%%%%%%%%%%

Let's approach the concept of environmental impact with a simple analogy: suppose you want---whatever the reason---to dig a hole in your well-grassed garden {\color{blue}[\stepcounter{slide}Slide \arabic{slide}]}. By doing it, you reduce the available surface to dig another hole in your garden. You consume a specific resource which is the ``area of nicely grassed ground". This resource becomes more scarce than it was before. Digging a hole also inevitably creates a heap somewhere else, further reducing the resource ``area of nicely grassed ground" of an additional amount (the surface of the heap base). 

In this analogy, the hole stands for the \emph{consumption}*, the heap stands for the \emph{pollution}* and both together stand for the \emph{depletion of a resource}* (the garden area). Resource depletion refers to a decrease in the available stock of resources---to the destruction of a bit of what we called earlier the ``environmental capital''. Environmental impact is the marginal contribution of a process to this depletion due to the direct consumption of resources or through the emission substances leading to its pollution. It is the combination of marginal environmental impacts generated by a large amount of diverse activities (like commuting in London) which generates pollution. 

This stated, let's now have a closer look at the kind of resources that can be depleted by consumption and what kind of pollution can contribute to this depletion.

\subsubsection{Resource consumption, depletion and restoration}
\label{sec:depletion}
Consumption either implies the destruction of a resource or its dispersion to an extent which hinders further use. An example of dispersion is the use of precious metals in semiconductors: extracting the ores, refining the metals, embedding them in products and using these products does not obliterate the chemical elements, it just leads to their geographic dispersion which ultimately makes them unavailable for further industrial use. An example of resource destruction is the combustion of oil-based fuels: the resource disappears as it is transformed into something else (CO\textsubscript{2}, among others). Resources ongoing destructive processes may further either be renewable or non-renewable. Those which are renewable can be restored, either through natural cycles (like clean freshwater) or through human intervention (like food). In this case, a depletion in the overall stock happens when the destruction of resources is quicker than the ability of the restoring mechanism to replenish the stock. The destruction of non-renewable resources can neither be restored by natural cycles nor by human intervention. Once a species is extinguished, once nuclear or fossil fuel is burnt, there is no comeback.

Excessive consumption is already or is about to become a critical issue for a large variety of economic sectors {\color{blue}[\stepcounter{slide}Slide \arabic{slide}]}. The remaining reserves of some precious metals like gallium and arsenic (used in semiconductors), indium and silver (used in photovoltaic panels), as well as gold and silver (used in electronic circuits) may be depleted in 5 to 50 years if consumption and disposal continues at present rate \cite{dodsonElementalSustainabilityTotal2012}. Those of uranium (used for nuclear energy production) as well as cadmium and nickel (used in batteries) may be depleted within 50 to 100 years (\emph{ibid.}). The depletion of other resources like oil, coal or wild fish is a well-known issue often covered in the media. The depletion of phosphate rock is another less covered ongoing issue threatening agriculture worldwide \cite{cooperFutureDistributionProduction2011}.

\subsubsection{Pollution}
\label{sec:pollution}
Pollution is defined as the introduction of physical (e.g. radiation, chemical substances) or biological (e.g. manure, seeds) agents into an ecosystem to an amount it cannot be absorbed, hence leading to its adverse disruption {\color{blue}[\stepcounter{slide}Slide \arabic{slide}]}. Pollution can be either natural (like the disastrous \href{https://en.wikipedia.org/wiki/1980_eruption_of_Mount_St._Helens}{eruption of Mount St. Helens} in 1980 and reducing hundreds of square miles to wasteland) or anthropogenic (like the ``\href{https://en.wikipedia.org/wiki/Smog}{smog}" affecting the health of citizen in urban environments). It can either be accidental (like the \href{https://en.wikipedia.org/wiki/Fukushima_Daiichi_nuclear_disaster}{Fukushima nuclear disaster}) or chronic (like the smog, once again). Its causes can either be local (like in the case of the \href{https://en.wikipedia.org/wiki/Hinkley_groundwater_contamination}{Hinkley groundwater contamination}) or dispersed (like the \href{https://en.wikipedia.org/wiki/Marine_debris}{omnipresence of long-lasting waste in oceans}). Among the most often considered pollutions of the anthropogenic, chronic and dispersed type are {\color{blue}[\stepcounter{slide}Slide \arabic{slide}]}:
\begin{itemize} % add examples or news for each of the examples (mer de plastic en espagne, protocole de montreal...)
	\item \href{https://en.wikipedia.org/wiki/Eutrophication}{eutrophication}: excessive growth of algae or plants in a body of water resulting from the excessive intake of nutrients such as nitrates or phosphates.
	\item \href{https://en.wikipedia.org/wiki/Acid_rain}{acid rains}: unusual acidity of rainfalls due to emissions of sulfur dioxide and nitrogen oxyde in the air
	\item \href{https://en.wikipedia.org/wiki/Ozone_depletion}{ozone hole}: depletion of the stratospheric ozone layer at Earth's poles due to the emission of ozone depleting substances such as CFCs, leading to a reduced share of UV radiation dispersed by the ozone layer. 
	\item \href{https://en.wikipedia.org/wiki/Smog}{smog}: noxious smoky fog hitting dense urban areas, composed of airborne particles and chemical compounds generated by combustion and leading to respiratory issues. 
	\item \href{https://en.wikipedia.org/wiki/Global_warming}{global warming}: rise in the average temperature of the Earth's surface resulting from an increased atmospheric greenhouse effect due to the massive airborne release of greenhouse gases such as carbon dioxide and or methane
	\item \href{https://en.wikipedia.org/wiki/Light_pollution}{light pollution}: high concentration of misdirected light at night in densely populated areas which may cause diffuse and chronic adverse effects on human health and disrupts wild life (e.g. bird migration) 
	\item \href{https://en.wikipedia.org/wiki/Land_use}{land use}: anthropogenic transformation of the natural terrestrial environment into functional spaces, leading to various adverse effects such as reduction of biodiversity, soil erosion, and water quality reduction. 
\end{itemize}
These examples show that there isn't \emph{one} pollution but rather a variety of ways ecosystems can be affected by anthropogenic activities. Different effects may even influence each other and be produced by the same agents, like the ozone depletion and global warming having complex interrelations and both being influenced by CFC emissions. Pollution may also affect complex sets of resources in a way that is not always understood, like global warming not only affecting weather conditions, but fresh water stock, emerged land, biological stock etc...

\subsubsection{Measuring environmental impacts}
\label{sec:context}
Unfortunately, there is no global reference of resources that need to be preserved from depletion. One reason is the absence of agreement on what is to be considered as an important resource. Some people would have directly thought about the well-being of bugs and worms in the garden analogy, others wouldn't. Opinions may differ whether the well-being of bugs is important, whether the available garden area is more important than the well-being of bugs, or whether these things are important at all. This also illustrates that there is no consensus on the relative importance of resources. Is it worse to contribute to ozone layer depletion or to the disruption of bee colonies? {\color{blue}[\stepcounter{slide}Slide \arabic{slide}]} All this is the expression of ethical values underpinning different positions in the debate between strong and weak sustainability. Another reason is the partly unequal global distribution of resources, making them differently critical for local populations: some regions have enough fresh water not to care about it, some countries have enough oil not to care about energy. A third reason is the absence of agreement on mechanisms to allocate critical resources, which is an utterly political topic. 

There is no global reference of pollutions that need to be mitigated as well. One reason is that some of them are localized, such as smog and light pollution only affecting urban areas. Whether pollution is considered as detrimental also depends on the perceived criticality of the resources involved in the affected ecosystem---and we have seen that there is no global consensus on that. Nevertheless, there are examples of measures to mitigate specific pollutions at global level, such as the \href{https://en.wikipedia.org/wiki/United_Nations_Framework_Convention_on_Climate_Change}{United Nations Framework Convention on Climate Change}, or the \href{https://en.wikipedia.org/wiki/Montreal_Protocol}{Montreal Protocol on Substances that Deplete the Ozone Layer}. A larger number and variety of agreements can be found at regional level, like the legally binding \href{https://en.wikipedia.org/wiki/European_emission_standards}{European emission standards for new vehicles} or more local levels. 

In spite of this lack of consensus, different approaches have been developed to define indicator sets or aggregated indicators of environmental impacts, such as \href{https://www.pre-sustainability.com/download/EI99_Manual.pdf}{EI99} \cite{consultants2000eco}, Impact 2002+ \cite{jollietIMPACT2002New2003}, ILCD 2011 or \href{https://www.leidenuniv.nl/cml/ssp/publications/recipe_characterisation.pdf}{ReCiPe 2008} \cite{goedkoop2009recipe} {\color{blue}[\stepcounter{slide}Slide \arabic{slide}]}. These indicators allow masking the complexity of environmental impacts behind a single or a few summarizing figures. They are useful for practitioners wanting to take environmental impact into account in decision making without being environmental experts or needing to take position on the relative importance of environmental impacts on behalf of their customers. Nonetheless, these indicators cannot account for the whole complexity of eco-systemic mechanisms. They are based on simplified models and ethical assumptions. Practitioners still need to make a choice between these indicators and apply them, which still requires some expertise.

\subsubsection{About irreversibility, or why less is better}
\label{sec:irreversibility}
Thermodynamic delivers us nice concepts to bypass this difficulty and to find simpler concepts to lean upon. In thermodynamical terms, depletion of resources can be termed as creation of \emph{entropy}. And the second law of thermodynamics tells us an interesting thing about entropy: in a closed system, it inescapably increases.

Let's apply this to the garden analogy. Suppose now you don't need the hole anymore and want to put everything back in place. You can restore the previous situation by putting the heap back in the hole. The result will however only \emph{approximate} the original situation: you will still have a discontinuity in the grass where the hole was and eventually a damaged grass where the heap was. Restoring the soil and grass in \emph{exactly} the same condition would require way more time and energy than what was necessary to dig the hole (e.g. you will eventually need to grow new grass). Because, in thermodynamical terms, what you want to do is to remove from the system the entropy you added by digging the hole. Removing entropy from a closed system requires creating elsewhere more entropy than the amount you want to remove. The domain of water management delivers another illustration of this principle of irreversibility {\color{blue}[\stepcounter{slide}Slide \arabic{slide}]}: mixing freshwater with any kind of liquid pollutant is easy and barely requires energy. But reversing the process, i.e. separating the liquids, is difficult and may require a lot of energy. What thermodynamics says, in other words: the energy required to mess up something is lower than those required to tidy it up.

Another interpretation of the second law of thermodynamics is that creating order somewhere is inescapably bond to creating more disorder somewhere else. The more organized you want the matter to get, the more disorder you have to create somewhere else. An illustration of this is provided by the semiconductor industry: the production of a 2g microchip, which embeds a fairly high organisation of matter requiring high material purity, requires more than 1.7kg of matter \cite{williamsKilogramMicrochipEnergy2002}. In other words, only 1.1\% of all matter involved in the production process ends up in the final product, 98.9\% turns out to be waste. Creating a platinum ring requires refining approx. 100kg of ores \cite{erkmanVersEcologieIndustrielle2004}, building a car 70t \cite{janinDemarcheEcoconceptionEntreprise2000}, sending an SMS 600g \cite{federico2001mips}. Creating fancy things creates huge messes. 

Understanding environmental impact as entropy amounts to say that every activity has an environmental impact, which is irreversible \cite{jackson2013material}\textsuperscript{\color{Magenta}[key reading]}. Trying to restore some resources (to ``recycle'' them) would only make the whole set of resources worse. Consequently, the best way to avoid environmental impact is to reduce activity. Less is better. In other words, the more social value is created out of the minimum activity, the better. Sustainability is about reducing waste, understood in the sense of the consumption of a resource which is not bound to human satisfaction.

% end of subsection - time for take aways and exercise
{\color{PineGreen}
\setlength{\parskip}{1em}
{\emph{Take-aways of this subsection:
\setlength{\parskip}{0em}
\begin{itemize}
  \item The environmental issue considered in sustainability is \emph{resource depletion}*
	\item Resources can either be depleted by \emph{consumption}* or by \emph{pollution}*
	\item Humans decide upon the importance of resources, and humans don't always agree
	\item The \emph{environmental impact}* of a process is its marginal contribution to resource depletion
	\item Every activity has an environmental impact, even resource restoration activities
	\item Less is better
\end{itemize}
}}

\begin{comment}
\setlength{\parskip}{1em}
\emph{Exercise \stepcounter{exercise}\arabic{exercise}. In groups of 4 people, discuss which environmental impacts you consider as important, how your views may fit with your future job sheet and the local conditions you are living in. Try to mix nationalities by constituting the groups, this will increase the chances to have conflicting opinions and enhance discussion.}
\setlength{\parskip}{0em}
\end{comment}
}

\subsection{Eco-efficiency of products}
\label{sec:tbd}
%%%%%%%%%%%%%%%%%%%%%%%%%%%%%%%%%%%%%%%%%%%%%%%%%%%%%%%%%%%%%%%%%%%%%%%%%%%%%%%%%%%%%%%%%%%%%%%%%%%%

Talking about products, where are the environmental impacts coming from? When my TV is off, it does not do anything harmful to anybody, it is just an inert object. That is, the mere existence of products does not create environmental impacts. Rather, products are `middle things' between social needs (watching TV) and large and complex networks of \emph{activities} caused by the realisation of these needs: me having a TV was only possible because somebody \emph{fabricated} it; me \emph{watching} it requires somebody to \emph{make} electricity and \emph{bring} it to the socket my TV is plugged on; me ultimately \emph{disposing} the TV will need somebody to \emph{process} it so it does not come back to the environment and leak harmful substances, etc. As we said earlier: each activity has environmental impacts. The environmental impacts associated with products sums up to those of all activities which contributed to the delivery of its specific functionality. The set of all these activities is called the \emph{product life cycle}*. The specific functionality is called the \emph{functional unit}*. The ratio between the sum of all impacts created within the product life cycle and its functional unit is called the \emph{eco-efficiency}*. Improving the sustainability of a product amounts to improve its eco-efficiency, which requires analyzing both the product life cycle and the functional unit.

\subsubsection{Product life cycle}
\label{sec:plc}
The concept of product life cycle is generally represented as a circle of chronological stages {\color{blue}[\stepcounter{slide}Slide \arabic{slide}]}\footnote{not to be mistaken with the \href{https://en.wikipedia.org/wiki/Product_life-cycle_management_(marketing)}{product life cycle used in innovation management}}. Typical stages are:
\begin{itemize}
	\item Raw materials extraction: resources are taken from their natural state, eventually refined and transformed into semi-manufactured products (e.g. bauxite is excavated, processed to produce alumina, smelted to product raw metal aluminium blocks)
	\item Manufacturing: semi-finished products are turned into functional components with the help of diverse processes (milling, casting, etc.), which are ultimately assembled into finished products and packaged for distribution
	\item Transport and distribution: products are transported from point of production to point of use, eventually involving diverse intermediaries (retailers, warehouses, etc.)
	\item Use: products deliver their functionality for the end-user, eventually in combination with other products or consumables (ink cartridges and electricity), leading to product fatigue, eventually requiring maintenance, reparation, and overhaul
	\item End-of-life: discarded products can be repaired, remanufactured, their materials recycled, hence avoiding the need to extract raw materials for future product generations, or sent to landfill.
\end{itemize}

This representation is simplistic in the sense that it suggests that a product life cycle is either a linear process with a single entry point (raw material extraction) and end point (landfill) or a perfectly circular process. In reality, the product life cycle is rather a complex network of interconnected activities taking inputs from different entry points and producing different products as well as more or less wished by-products {\color{blue}[\stepcounter{slide}Slide \arabic{slide}]}. Each of these activities involve what are called \emph{elementary} and \emph{non-elementary flows}*. Elementary flows are either direct inputs from natural resources (e.g. O\textsubscript{2} intake in combustion) or direct outputs to the environment (e.g. consequent NO\textsubscript{x} emission back in the air). Non-elementary flows are anthropogenically produced things, i.e. which are the product of human intervention (e.g. diesel, TV). The environmental impact of a product is the sum of all elementary flows of all processes involved in its life cycle. The example of all non-elementary flows involved in the production of bread shows how complicated life cycles can be even for simple products {\color{blue}[\stepcounter{slide}Slide \arabic{slide}]} \cite{anderssonLifeCycleAssessment1999}.

\subsubsection{Functional unit}
\label{sec:fu}
The functional unit is a measure of the amount of functionality delivered by a product. This amount is defined in both qualitative and quantitative terms: \emph{what} is delivered by the product and \emph{how much} of it it can deliver. For example, the functional unit of a pen is not only given by its specific function---which is to draw a visible line on a predefined range of materials---but also by the total length it can draw (e.g. 2km) {\color{blue}[\stepcounter{slide}Slide \arabic{slide}]}. 

The quantitative part of the functional unit is important because it delivers the only reference allowing to compare the eco-efficiency of products objectively. If two pens deliver the same functional quality (same line characteristics such as darkness and regularity), have the same environmental impacts, but one of them can draw half the length of the other, then it is clear which of them is the most eco-efficient. It is more difficult---and often does not make any sense---to compare the environmental impacts of two products having diverging functions. What if two pens deliver lines of different thicknesses? What creates the most usage value: a car or a cell phone?

The extent to which the potential quantity of functionality a product can deliver is actually realised depends on usage patterns. This encompasses on the one side the appropriateness of the usage but also on the willingness of the user to \emph{exhaust} the potential of a product. A typical example of this phenomenon is the cell phone: there is a high probability that you have one of them sleeping in one of your drawers at home \cite{hanson2014s} {\color{blue}[\stepcounter{slide}Slide \arabic{slide}]}. These products can still deliver their functionality, but they have become obsolete, either for structural reasons (switch from 3G to 4G) or as a result of changing fashion (new iPhone coming out), among other possible reasons. Therefore, it is not only interesting to have a look at the potential quantity of functionality delivered by a product, but also at the extent to which users turn this potential into reality. 

\subsubsection{The role of design in product eco-efficiency}
\label{sec:productdesign}
There is a word of saying that 80\% of the costs of a product are engaged by its design, the rest being influenced by the later intervention of other stakeholders of the product life cycle, like the manufacturer and the user \cite{mcaloone2009environmental}\textsuperscript{\color{Magenta}[key reading]} {\color{blue}[\stepcounter{slide}Slide \arabic{slide}]}. Another common place is that the cost of changes increase with time along the product development process \cite{luttroppEcoDesignTenGolden2006a}\textsuperscript{\color{Magenta}[key reading]}. The same applies to the environmental impacts. This is due to the fact that the essence of design is to make \emph{prescriptive} decisions. When you design a product, you decide how the product should be manufactured (because you define materials, shapes, and assembly principles) and used. You therefore decide upon the cost and the environmental impacts of manufacturing and use. Stakeholders of the product life cycle still have a certain degree of freedom, yet constrained by your decisions. Similarly, decisions taken at the early design stages constraints the degrees of freedom at later stages. Hence, the best time to think about eco-efficiency is in the early design stages. 

% end of subsection - time for take aways and exercise
{\color{PineGreen}
\setlength{\parskip}{1em}
{\emph{Take-aways of this subsection:
\setlength{\parskip}{0em}
\begin{itemize}
  \item The sustainability of a product is given by its \emph{eco-efficiency}*
  \item Eco-efficiency is the ratio between two quantities: impact and function
  \item The quantity of impacts is given by the sum of all \emph{elementary flows}* involved in the \emph{product life cycle}*
  \item The quantity of function is given by the \emph{functional unit}*
  \item You can only compare the eco-efficiency of products having the same function
  \item The earlier you think about sustainability in the product development process, the better.
\end{itemize}
}}

\begin{comment}
\setlength{\parskip}{1em}
\emph{Exercise \stepcounter{exercise}\arabic{exercise}. Try to define in which terms the eco-efficiency of a car could be defined. How does the product life cycle looks like? What are the major elementary flows? What is the functional unit of a car? Where could be impacts avoided or functionality gained?}
\setlength{\parskip}{0em}
\end{comment}
}

\section{Eco-design}
\label{sec:ecodesign}
%%%%%%%%%%%%%%%%%%%%%%%%%%%%%%%%%%%%%%%%%%%%%%%%%%%%%%%%%%%%%%%%%%%%%%%%%%%%%%%%%%%%%%%%%%%%%%%%%%%%
%%%%%%%%%%%%%%%%%%%%%%%%%%%%%%%%%%%%%%%%%%%%%%%%%%%%%%%%%%%%%%%%%%%%%%%%%%%%%%%%%%%%%%%%%%%%%%%%%%%%

There is no magical formula to increase the eco-efficiency of products; no systematic method leading invariably to positive results. The essence of design is to develop solutions for open, complex and ill-defined problems. And for those problems, there is generally no systematic method designers can be provided with in order to derive optimal solutions. Nonetheless, like any other design parameter, eco-efficiency can be approached through a simple rational process in three steps:
\begin{itemize}
	\item Setting up objectives: what do I want to do and what are my motivations for that?
	\item Assessing the initial situation: how far am I from my objectives?
	\item Looking for solutions: what can I do to reach my objectives?
\end{itemize}

\subsection{Setting up objectives}
\label{sec:objectives}
%%%%%%%%%%%%%%%%%%%%%%%%%%%%%%%%%%%%%%%%%%%%%%%%%%%%%%%%%%%%%%%%%%%%%%%%%%%%%%%%%%%%%%%%%%%%%%%%%%%%

Successful integration of eco-efficiency as a product requirement in design requires identifying an underlying motivation (why should we do that?) and a clear set of quantified objectives (what do we aim for?). In the first subsection, we deliver an overview of the different possible motivations for a company to engage into eco-design. In the second section, we review the different schemes a company can lean upon to identify sound objectives.

\subsubsection{Motivations for eco-design}
\label{sec:motivations}
There is a wide range of reasons why a company may have interest in engaging in eco-design. Among these:
\begin{itemize}
	\item External factors:
	\begin{itemize}
		\item Legislation. While environmental regulation generally focus on companies and factories, some of them, especially in the EU, focus on products. This is the case of the \href{https://eur-lex.europa.eu/legal-content/EN/ALL/?uri=CELEX:32009L0125}{EU directive 2009/125/EC} called ``ErP'' (Energy-related Products), also known as the \href{https://en.wikipedia.org/wiki/European_Ecodesign_Directive}{``European eco-design directive''}. This directive sets targets for in-use energy consumption for of energy-using products (like home appliances such as \href{https://ec.europa.eu/energy/en/topics/energy-efficiency/energy-efficient-products/fridges-and-freezers}{fridges}) as well as energy-related performance targets for products having influence on energy consumption (like \href{https://ec.europa.eu/energy/en/topics/energy-efficiency/energy-efficient-products/tyres}{tyres}) {\color{blue}[\stepcounter{slide}Slide \arabic{slide}]}. Electric and electronic products are also indirectly targeted by the \href{https://eur-lex.europa.eu/legal-content/EN/TXT/?uri=CELEX:32012L0019}{EU directive 2012/19/EU} called ``WEEE'' (Waste Electrical and Electronic Equipment). This directive requires producers of electric and electronic product to participate in the establishment of recycling efforts. While this directive does not directly set requirements to products, it creates incentives for producers to manufacture products which are easy to recycle. Compliance with these directives are required for the product to carry the \href{https://en.wikipedia.org/wiki/CE_marking}{CE marking}.
		\item Societal pressure. As eco-efficiency is increasingly being acknowledged by the general public as a valuable product feature, demand for explicitly labeled 'eco-friendly' products increases and those for products with negative environmental image decreases. Today's marketing teams have all in mind the story of large companies like Nike or Gap whose questionable behavior (e.g. bad working conditions, child labor) have been publicly exposed in works like \href{https://en.wikipedia.org/wiki/No_Logo}{\emph{No Logo} from Naomi Klein} in the 90's, leading to negative customer reactions {\color{blue}[\stepcounter{slide}Slide \arabic{slide}]}. Today, the reputation of a large company is constantly kept under scrutiny, leaving companies little space for neglecting their social responsibility. At the same time, compliance with sustainable practice became a competitive argument. This is shown by the emergence of a large variety of product labels related to sustainability we will look at in more detail in the next section (\ref{sec:standards}).
	\end{itemize}
	\item Internal factors:
	\begin{itemize}
		\item Cost reduction: As we stated in the previous section: less is better. Eco-design is about reducing waste, i.e. the consumption of a resource which does not satisfy any need. And if it does not satisfy any need, there is a high probability you won't be able to charge for it. And if you can't charge for it, it is a cost for you. Eco-design can be a strategy to hunt unnecessary costs and to increase the economic viability of a company {\color{blue}[\stepcounter{slide}Slide \arabic{slide}]}. In this sense, eco-design is close to management approaches like \href{https://en.wikipedia.org/wiki/Lean_manufacturing}{lean manufacturing} which are targeted at the systematic identification and elimination of unnecessary efforts. 
		\item Internal momentum: eco-design can participate to the general effort of a company to establish a corporate culture {\color{blue}[\stepcounter{slide}Slide \arabic{slide}]}. Sustainability conveys positive value, which can act as a motivating factor for employees and support constructive mind sets. Because eco-design requires the involvement of all departments of a company, the establishment of an organisational structure to support eco-design can lead to more communication between departments and lead to positive side effects such as the early identification of mistakes and misunderstandings. 
	\end{itemize}
\end{itemize}
While external factors give the impression that sustainability is an additional constraint burdening companies, the internal factors show that engaging in eco-design can turn into net benefits. Eco-designed products are not necessarily more expensive, contrarily to the general belief.

\subsubsection{Where to search for help to define eco-design objectives?}
\label{sec:standards}
Section \ref{sec:sustainability} and especially \ref{sec:context} depicted a rather puzzling picture of sustainability as it stated that there is no global reference of environmental impacts which have to be cared for. Wile there is no generally applicable reference, there may be standards applying to specific geographic regions or to specific product branches. These standards discharge companies of the burden to define by themselves which environmental impacts they need to consider, which volume of impact is acceptable and how to measure it. They set criteria which are relevant for a given regional context or product, and eventually provide compliance thresholds. Let's have a look at some examples of these:
\begin{itemize}
	\item Regulatory requirements. The \href{https://eur-lex.europa.eu/legal-content/EN/TXT/?qid=1399998664957&uri=CELEX:02011L0065-20140129}{EU directive 2011/65/EU}``RoHS'' (Restriction of Hazardous Substances) restricts the use of ten especially noxious substances in electrical and electronic products, including lead, mercury, hexavalent chromium and specific flame retardants like phthalates. This directive provides a clear guidance, as it sets the list of products affected by these restriction, the list of substance which are restricted as well as the maximal allowed thresholds.
	\item Voluntary governmental ecolabels. The \href{http://ec.europa.eu/environment/ecolabel/}{EU-Ecolabel} is a voluntary label provided by the European Commission. It states a list of eco-efficiency-related product criteria for a large range of products categories such as televisions, floor coverings, paper, toilets, among others. For example, the criteria list for the \href{https://eur-lex.europa.eu/LexUriServ/LexUriServ.do?uri=OJ:L:2013:353:0053:0063:EN:PDF}{category ``imaging equipment''} sets thresholds for indoor air emissions, requires printers to offer printing more than one page on one sheet of paper and OEMs to provide a warranty of minimum 5 years, among others {\color{blue}[\stepcounter{slide}Slide \arabic{slide}]}. If your product does not fit in one of the provided categories, you can create one by approaching the Ecolabelling Board to participate in a standardisation process and to define new criteria. 
	\item Branch specific ecolabels. \href{http://greenelectronicscouncil.org/epeat/epeat-overview/}{EPEAT} (Electronic Product Environmental Assessment Tool) is a US and non-profit initiative delivering an eco-label for B2B electronic products. Specific criteria for each product category are based on international standards, like the \href{https://ieeexplore.ieee.org/document/8320570/}{IEEE 1680.1™ – 2018 Standard for Environmental and Social Responsibility Assessment of Computers and Displays}. Based on a point system, it delivers three levels of compliance: bronze, silver and gold. The textile branch gives us two other examples of branch specific labels (B2C, this time): \href{https://www.oeko-tex.com/}{Oeko-tex} and \href{https://www.global-standard.org/}{The Global Organic Textile Standard (GOTS)}.
	\item Branch and impact specific ecolabels. \href{https://www.energystar.gov/}{Energy star} is a US government-led voluntary program to promote energy efficiency. It delivers a label to products whose energy consumption is lower than a maximal allowed value for their product category. This maximal value is given by a mathematical function of the product features. For example, for monitors, the higher the resolution and the area of a monitor are, the higher the energy allowance is. The complete formula for defining the maximal allowed energy consumption of a monitor can be seen \href{https://www.energystar.gov/products/office_equipment/displays/displays_key_product_criteria}{here}.
\end{itemize}
The three first types of standards are multi-criteria and cover all relevant impacts of a product category involved in the whole product lifecycle. These criteria are developed in standardisation processes involving governmental bodies, companies and NGOs, and are backed on detailed environmental studies. They are therefore a good starting point to have an idea of the relevant environmental impacts of a product. The last type of standards focus on one specific environmental impact, which may not always be the most relevant one for a given product category. The corresponding criteria should therefore not be considered as a proxy for the total environmental impact of a product. 

There are a lot of ecolabels {\color{blue}[\stepcounter{slide}Slide \arabic{slide}]} and there may be different online databases referencing them, like \href{http://www.ecolabelindex.com/ecolabels/?st=country,gb}{this one}. Browsing the criteria of the labels applying to a product category of interest is a good starting point for an eco-design initiative. In case the product isn't covered by any label, it is still possible to perform an environmental assessment, which is the topic of the next section. 

% end of subsection - time for take aways and exercise
{\color{PineGreen}
\setlength{\parskip}{1em}
{\emph{Take-aways of this subsection:
\setlength{\parskip}{0em}
\begin{itemize}
	\item Eco is not more expensive: there are good reasons for a company to involve in eco-design
  \item Legislation and eco-labels are good sources of information for jumping in eco-design
\end{itemize}
}}

\begin{comment}
\setlength{\parskip}{1em}
\emph{Exercise \stepcounter{exercise}\arabic{exercise}. Make an internet research to find out which environmental standards would apply for to a fridge. Can you think of any way to outperform these standards?}
\setlength{\parskip}{0em}
\end{comment}
}

\subsection{Assessing the initial situation}
\label{sec:envAss}
%%%%%%%%%%%%%%%%%%%%%%%%%%%%%%%%%%%%%%%%%%%%%%%%%%%%%%%%%%%%%%%%%%%%%%%%%%%%%%%%%%%%%%%%%%%%%%%%%%%%

This stage is about identifying the environmental \emph{hotspots}* of a product, i.e. which activities of the product life cycle deliver the largest share of all the environmental impacts of the product and are consequently to be focused on in priority. Insights may have already been gained from standards in the previous stage \ref{sec:standards}, since product criteria are generally set to cover the environmental hotspots. However, more detailed and product-specific insights may be delivered by the direct application of \emph{environmental assessment}* methods. The queen of all environmental assessment methods is \emph{LCA}* (life cycle assessment) because of its rigour and precision. LCA is however an overly time consuming method requiring significant expertise. Environmental assessment can be approached through streamlined methods, like the MECO matrix. 

\subsubsection{Life Cycle Assessment (LCA)}
\label{sec:lca}
Life cycle assessment is an internationally recognized method whose principles are recorded in the standard \href{https://www.iso.org/en/standard/37456.html}{ISO 14040}. It basically implies four steps:
\begin{enumerate}
	\item Defining the boundaries of the system, that is, what belongs to the product life cycle and what can be left out. In practice, all economic activities are interconnected, so that it is not possible to make a clear cut between what contributes to the delivery of a product and what does not play a role. For example, to dig the hole in your garden, you need a spade. A certain percentage of the environmental impacts of the spade can be allocated to the hole, because you bought the spade to dig holes in general and \emph{this} hole in particular. And to make the spade, you needed a machine. A certain percentage of the environmental impact of the machine can be allocated to the spade and to the hole, because the reason to be of this machine was partly to produce a spade so you can dig a hole. This causal propagation can be infinitely extended like a fractal picture {\color{blue}[\stepcounter{slide}Slide \arabic{slide}]}, but would not lead to any additional useful insights. The objective of this step is to find where to stop it.
	\item Performing an \emph{inventory} of all elementary flows involved in all activities of the product life cycle. All inputs and emissions are identified by a type of substance and a quantity (e.g. 200g of NO\textsubscript{X}) and are recorded in a large spreadsheet {\color{blue}[\stepcounter{slide}Slide \arabic{slide}]}. Quantities of the same substances are added up. 
	\item Computing the impact associated with these inputs and emissions using environmental indicators (some of which have been introduced in section \ref{sec:context}). These indicators associate quantified specific environmental effects to the substances identified in the previous stem. For example, the impact of a substance regarding climate change is given by the Global Warming Potential (GWP). \href{http://www.ghgprotocol.org/sites/default/files/ghgp/Global-Warming-Potential-Values\%20\%28Feb\%2016\%202016\%29_1.pdf}{The global warming potential of CO\textsubscript{2} is 1 and those of SF\textsubscript{6} is 22,800}. If you have in your inventory emissions of 200g of CO\textsubscript{2} and 2g of SF\textsubscript{6}, you get a total GWP of 200x1 + 2x22,800 = 23,000. This can be made for different environmental indicators {\color{blue}[\stepcounter{slide}Slide \arabic{slide}]}, whose values can be eventually summed up to generate an unique environmental impact value.
	\item Interpreting the result and identifying the hotspots. Displaying the results in flow diagrams {\color{blue}[\stepcounter{slide}Slide \arabic{slide}]} or \href{https://en.wikipedia.org/wiki/Sankey_diagram}{Sankey diagrams} help to find which processes in the life cycle lead to the largest part of environmental impacts. 
\end{enumerate}

The major drawback of life cycle assessment is that it requires information which is only available when a product is fully defined. As a consequence, it cannot be used in early design stages, where all possibilities are still open. Nonetheless, in most cases, a new product is either the redesign of an already existing product or a combination of existing technologies. It is therefore possible to base on the LCA of an existing product in order to define which hotspots are to be addressed in the design of a new generation of this product. 

\subsubsection{MECO matrix}
\label{sec:streamlined}

LCA remains a time-consuming tool reserved for experts and does not fit with the time line of early design stages. An alternative to LCA is to use qualitative environmental assessment tools such as the \emph{MECO matrix}*. MECO stands for Materials, Energy, Chemicals and Other. It takes the form of a 5x4 matrix where the columns are the 5 life cycle phases `materials', `manufacture', `distribution', `use' and `disposal', and the rows are the 4 MECO environmental aspects `materials',  `energy', `chemicals' and `others' {\color{blue}[\stepcounter{slide}Slide \arabic{slide}]}. This matrix is meant to be filled by thinking about each of the environmental aspects across each phase of the product life cycle.

Because it is qualitative, this method is particularly adapted for early design stages where no quantitative information is available on the product. Because it only requires a single sheet of paper, it is particularly adapted for group discussions and quick decision making. However, as it is a qualitative approach it is more subjective and less systematic than LCA, so it may overlook some important issues.

% end of subsection - time for take aways and exercise
{\color{PineGreen}
\setlength{\parskip}{1em}
{\emph{Take-aways of this subsection:
\setlength{\parskip}{0em}
\begin{itemize}
	\item Environmental assessment helps identifying \emph{hotspots}* to prioritize in an eco-design project
	\item \emph{LCA}* is the most rigourous environmental assessment method but is very time consuming
	\item In early design stages, LCA can be substituted to streamlined assessment methods like the \emph{MECO matrix}*
\end{itemize}
}}

\begin{comment}
\setlength{\parskip}{1em}
\emph{Exercise \stepcounter{exercise}\arabic{exercise}. In groups of 4, on a white board or a sufficiently large sheet of paper, draw the MECO matrix and try to fill it out for the product 'smart phone'}
\setlength{\parskip}{0em}
\end{comment}
}

%%%%%%%%%%%%%%%%%%%%%%%%%%%%%%%%%%%%%%%%%%%%%%%%%%%%%%%%%%%%%%%%%%%%%%%%%%%%%%%%%%%%%%%%%%%%%%%%%%%%
\subsection{Looking for solutions}
\label{sec:solutions}
Once the environmental hotspots of the product are known, it is time to look for alternative designs potentially offering an increased functionality/impact ratio. For this, you can either rely on \emph{design for the environment (DfE) guidelines}* an on \emph{eco-ideation mechanisms}*. DfE guidelines provide good ideas to alter an already existing design. They support incremental product improvement. Eco-ideation mechanisms are creativity triggers to find your own original solutions. They support radical product innovation. 

\subsubsection{DfE guidelines}
\label{sec:guidelines}
In the absence of magic formula to design sustainable products, guidance can be given to designers in the form of guidelines, i.e. strategies, or `rules of thumb', which have proven to lead towards satisfactory solutions but which are not guaranteed to be optimal. Academia has produced large numbers of these rules of thumb by analyzing existing designs. We can mention ``10 golden rules'' of eco-design by Luttropp and Lagerstedt \cite{luttroppEcoDesignTenGolden2006a}\textsuperscript{\color{Magenta}[key reading]}. More detailed and comprehensive guidelines are given by Telenko et al. \cite{telenkoCompilationDesignEnvironment2016a}\textsuperscript{\color{Magenta}[key reading]}, whose 76 guidelines are classified in subsets specifically targeting environmental aspects in the product life cycle {\color{blue}[\stepcounter{slide}Slide \arabic{slide}]}. Another source of DfE guidelines is the online tool \href{http://pilot.ecodesign.at/pilot/ONLINE/ENGLISH/INDEX.HTM}{ECODESIGN PILOT}. It provides an interactive environment to navigate among eco-design guidelines and find the most appropriate ones for a given hotspot. In the following paragraph, we introduce some examples of DfE guidelines targeting at the product end-of-life (design for multiple life cycles) and use phase (design for energy efficiency).

\paragraph{Design for multiple life cycles.}
\label{sec:DfR}
This approach is particularly relevant for products whose major environmental impacts lie in the raw material extraction phase or whose content may be harmful once released into the environment. There are three options for material recovery at end-of-life:
\begin{enumerate}
	\item Reuse. After eventual repair or refitting, the disposed product can be reused by another user. Eventually, the product can be dismantled and its parts reused for alternative purposes. This is then called \emph{up}cycling. 
	\item Remanufacturing. Disposed products of a same model are collected and dismantled to separate parts. Parts are tested and sorted. Those which are still fully functional are reintroduced in the production process. The others undergo an alternative end-of-life route.
	\item Recycling. Mixed disposed products are collected and dismantled or shredded to separate materials. Materials are sorted and introduced in a recycling process. In most cases, recycling amounts to \emph{down}cycling in the sense that the recycled material has not exactly the same properties than the original material (e.g. polymers in plastics get shorter).
\end{enumerate}
These recovery methods are presented in the order of increasing additional impact/entropy and decreasing preservation of the value added {\color{blue}[\stepcounter{slide}Slide \arabic{slide}]} \cite{mihelcicSustainabilityScienceEngineering2003}. Recycling amounts to destructing the geometrical shape of a product and turning it to another half-finished product (e.g. plastic pellets). Both the destruction of the value added by forming processes and the application of a new forming process create impacts. Recycling only makes sense when these additional impacts are lower than those of forming products out of virgin material. Remanufacturing intends to use parts \emph{as is}: it keeps the value added in the part geometry. Therefore, it potentially leads to lower impacts and should be preferred to recycling. Upcycling potentially leads to even lower processing and should be preferred to remanufacturing. However, upcycling implementation at industrial scale is impeded by the difficulty of finding stable and localised waste flows {\color{blue}[\stepcounter{slide}Slide \arabic{slide}]}.

The following two boxes give some examples of design for remanufacturing and design for recycling guidelines. More can be found in Go \emph{et al.} \cite{goMultipleGenerationLifecycles2015a}.

\begin{framed}
\footnotesize
\paragraph{Examples of Design for Remanufacturing guidelines}
\begin{itemize}
	\item Modularize the product to group parts depending on their own life cycle. 
	\begin{itemize}
		\item Group parts per amount of value added. Group low value parts together and high value parts together. This will reduce the number of required processes to dismantle high value parts. 
		\item Localize wear functions on easily detachable parts {\color{blue}[\stepcounter{slide}Slide \arabic{slide}]}. This reduces the volume of material needing replacement in the remanufacturing process.
		\item Make the most valuable parts or fragile parts appear first in the disassembly sequence to reduce the required disassembly time {\color{blue}[\stepcounter{slide}Slide \arabic{slide}]}.
	\end{itemize}
	\item Make the disassembly process easier. Reduce the level of intervention required to extract the parts to be reintroduced in the production process.
	\begin{itemize}
    \item Use easily reversible joinings. Prefer nuts and bolts to snap fits.
		\item Reduce the variety of joining techniques. The higher the number of different joining techniques, the higher the number of different disassembly processes is required, the more expensive disassmbly is. 
		\item Reduce the variety of directions of the joining techniques. Placing all entry points for disassembly on one side of the product avoids turning the product during the disassembly process and reduces the disassembly time.
	\end{itemize}
\end{itemize}
\end{framed}

\begin{framed} 
\footnotesize
\paragraph{Examples of Design for Recycling guidelines}
\begin{itemize}
	\item Choose materials which can be recycled. There is not necessarily a recycling process available for each material. Composite materials for example are difficult to recycle because there is no inexpensive way of either separating the constituents of the composites or reshaping composites for another usage. 
	\begin{itemize}
		\item Choose materials for which the existing recycling process is actually performed by some companies. That a process is available in theory does not mean it is actually performed by someone; some recycling processes may be at a protoype stage.
		\item Choose materials which can be sold after recycling, i.e. for which there is a market. It may be technically possible to recycle a product but there may not be any market for buying the recycling material. 
	\end{itemize}
	\item Make materials separable if they are not compatible. Recycling requires separating materials which cannot be recycled together. Separation includes detachment and sorting. 
	\begin{itemize}
		\item Choose materials which can be sorted by conventional separating processes (e.g. float-sink {\color{blue}[\stepcounter{slide}Slide \arabic{slide}]} or air separation for plastics, magnetic or induction separation for metals, among other techniques)
	  \item Avoid irreversible joining processes such as over-moulding. Even in shredding, over-moulding does not allow the separation of materials, whatever their size {\color{blue}[\stepcounter{slide}Slide \arabic{slide}]}.
	\end{itemize}
	\item Avoid the use of incompatible materials, i.e. materials that disturbs the recycling process of another material {\color{blue}[\stepcounter{slide}Slide \arabic{slide}]}.
\end{itemize}
\end{framed}

\paragraph{Design for energy efficiency.}
\label{sec:DfEE}
Design for energy efficiency is relevant for products whose major environmental impacts are generated through energy consumption in the use phase. The energy efficiency of industrial products depends on three factors: the theoretical minimum required to deliver the function, the intrinsic losses of the product and the user-induced losses \cite{eliasUserefficientDesignReducing2011} {\color{blue}[\stepcounter{slide}Slide \arabic{slide}]}. While the physical minimum energy to perform a given work cannot be changed, the volume of work maybe something that can be worked on. Let's take the example of using a kettle to make tea {\color{blue}[\stepcounter{slide}Slide \arabic{slide}]}. Most kettles are designed to shut off once water boils. However, most teas require the water to be between 70 and 85 \textdegree C. Reducing the target temperature reduces the amount of work to be delivered by the kettle and increases the energy efficiency. User-induced losses can be addressed as well in this example: how often do you heat exactly as much water as you need? Often, users would tend to boil too much water to be sure they have enough. Giving the user the possibility to measure the exact amount of water they need can avoid energy losses. The following box provides some examples of design for energy efficiency guidelines. More design for energy efficiency can be found in Telenko \emph{et al.} \cite{telenkoCompilationDesignEnvironment2016a} as well as Bonvoisin \emph{et al.} \cite{bonvoisin2010design}.

\begin{framed} 
\footnotesize
\paragraph{Examples of Design for Energy Efficiency guidelines}
\begin{itemize}
  \item In cases the amount of functionality wished by the user is variable, provide automatic or manual tuning capabilities so the product does not provide more than expected.
	\item Provide user with feedback on the current state of the process and eventually on how much resource is being consumed. This will allow them to adapt their demand. 
	\item Provide discrete quantities of resources to avoid over-use. 
	\item Schedule an automatic power down.
\end{itemize}
\end{framed}



% end of subsection - time for take aways and exercise
{\color{PineGreen}
\setlength{\parskip}{1em}
{\emph{Take-aways of this subsection:
\setlength{\parskip}{0em}
\begin{itemize}
	\item To improve the eco-efficiency of a product, you can rely on existing \emph{DfE guidelines}*.
	\item DfE guidelines provide ideas/examples/principles which can be adapted to your own product.
	\item DfE guidelines can be used for addressing specific hotspots such as energy consumption in use or recyclability. 
\end{itemize}
}}

\begin{comment}
\setlength{\parskip}{1em}
\emph{Exercise \stepcounter{exercise}\arabic{exercise}. In groups of 2, draw the MECO matrix of a disposable razor. In ECODESIGN PILOT look for the relevant DfE guidelines. Apply some of these guidelines to re-design the product.}
\setlength{\parskip}{0em}
\end{comment}
}

\subsubsection{Eco-ideation mechanisms}
\label{sec:ecoAsit}
%%%%%%%%%%%%%%%%%%%%%%%%%%%%%%%%%%%%%%%%%%%%%%%%%%%%%%%%%%%%%%%%%%%%%%%%%%%%%%%%%%%%%%%%%%%%%%%%%%%%
The eco-ideation mechanisms introduced by Tyl et al. \cite{tyl2016esm} are also rules of thumbs like the DfE guidelines. However, they are not directed towards a specific environmental aspect. Instead, they are meant to relax the constraints of the design problem, to challenge the design brief and to lead to more radical ideas. The eco-ideation mechanisms are meant to be used in creativity sessions such as brainstorming. 
\begin{itemize}
	\item \emph{Innovate with stakeholders.} What value is captured, destroyed or missed by the different stakeholders of the product life cycle (e.g. customers, business, environment and society)?
	\item \emph{Innovate through biomimicry.} How would the product look like if it was a natural organism or an ecosystem of physical (i.e. resources, energy) or informational flows?
	\item \emph{Innovate through sustainable mode of consumption.} To what extent the user could have a personal interest in making a sustainable use of the product?
	\item \emph{Innovate through Product Service Systems.} What about focusing on the service delivered by the product rather than on the product itself? Do we need to sell that product? (More about this in Lecture 9 ``systems thinking'').
	\item \emph{Innovate through territorial resources.} How could the local natural, industrial and social capitals can be used for making the product and how can the product contribute to building up these capitals?
	\item \emph{Innovate through circularity.} What could be done to make of the product life cycle a completely circular and closed system?
	\item \emph{Innovate through new technologies.} Are there alternative ways to manufacture products, from revival of traditional techniques to new production means such as additive manufacturing?
\end{itemize}

\section{Outlook: beyond eco-design}
\label{sec:SPD}
%%%%%%%%%%%%%%%%%%%%%%%%%%%%%%%%%%%%%%%%%%%%%%%%%%%%%%%%%%%%%%%%%%%%%%%%%%%%%%%%%%%%%%%%%%%%%%%%%%%%
%%%%%%%%%%%%%%%%%%%%%%%%%%%%%%%%%%%%%%%%%%%%%%%%%%%%%%%%%%%%%%%%%%%%%%%%%%%%%%%%%%%%%%%%%%%%%%%%%%%%
The aim of this section is to mention alternative, less established but interesting approaches to sustainable product design.

\paragraph{Social impacts and S-LCA.} In this lecture, we focused on the environmental dimension of sustainability. Therewith, we omitted that processes also have \emph{social} impacts which also are to be reduced. This is for example the approach of the \href{https://www.fairphone.com/en/}{FairPhone} project {\color{blue}[\stepcounter{slide}Slide \arabic{slide}]}, striving for a fair supply chain limiting dangerous working conditions and child labor. This is also the purpose of \emph{social life cycle assessment} (S-LCA), a promising method to assess the socio-economical aspects that ``directly affect stakeholders positively or negatively during the life cycle of a product'' \cite{andrews2010guidelines} {\color{blue}[\stepcounter{slide}Slide \arabic{slide}]}. This method applies the same basic principles as LCA---to compile interactions between processes and the environment and assess their impact---but on the social dimension of sustainability. It assesses social aspects using indicators such as ``health, autonomy, safety, security \& tranquility, equal opportunities, participation \& influence, resource (capital) productivity''. 

Why is this topic is not covered by this lecture? We saw in section \ref{sec:context} that there is always an element of subjectivity in assessing environmental impacts since there is no global reference of environmental resources that need to be preserved from depletion. This challenge is even more complex in the case of the social impacts, because different people and nations may value social aspects differently. For example, the notion of individuality may differ across societies, influencing the role of individuals, the meaning of social equity and citizen participation. It is therefore complicated to find indicators which are free from ethical views. In line with these difficulties, S-LCA is to date not as much established than LCA in industry but is developed enough to provide solid advice when applied with dedicated expertise. 

\paragraph{Charitable design.} Wide ranges of products are designed to fulfill charitable intentions. In contrast with S-LCA which is about reducing negative social side-effects of products, charitable design is meant to deliver products whose functionality creates outstanding positive social value. These products generally address what is called the ``base of the pyramid'' (BoP), referring to \href{https://en.wikipedia.org/wiki/Maslow\%27s_hierarchy_of_needs}{Maslow's (heavily contested) pyramidal representation of the hierarchy of human needs}. At the base of this hierarchy are the so-called ``physiological'' needs such as eating, drinking, breathing, sleeping, have a shelter. Designing for the base of the pyramid is designing products helping unfavored populations to satisfy these physiological needs. Charitable design is often used for means of development aid, as shown by recent and iconic products such as \href{https://gravitylight.org/}{GravityLight}, \href{http://literoflightswitzerland.org/}{LiterOfLight}, \href{http://faircap.org/}{Faircap} or even \href{https://www.flexipump.com/}{FlexiPump}, a project initiated here from a Bath MechEng student {\color{blue}[\stepcounter{slide}Slide \arabic{slide}]}. Charitable design can also be applied to inland issues, as brilliantly exemplified by the Bath company \href{https://designability.org.uk/}{designability} designing product for disabled people. 

Why this topic is not covered by this lecture? Designing products whose functionality deliver positive social value can be just considered as a good design practice---some kind of designer's version of the hypocratic oath, a swear not to design junk. One can consider that charitable design is just an (unfortunately not very profitable) niche market. Others may think that design for the BoP, as a part of development aid initiatives, is an attempt to project European values to other populations and is neo-colonialist. It belongs to every designer to take position in this ethical debate.


\section{Exercise}
\label{sec:exercise}
%%%%%%%%%%%%%%%%%%%%%%%%%%%%%%%%%%%%%%%%%%%%%%%%%%%%%%%%%%%%%%%%%%%%%%%%%%%%%%%%%%%%%%%%%%%%%%%%%%%%
%%%%%%%%%%%%%%%%%%%%%%%%%%%%%%%%%%%%%%%%%%%%%%%%%%%%%%%%%%%%%%%%%%%%%%%%%%%%%%%%%%%%%%%%%%%%%%%%%%%%
{\color{PineGreen}
Your are part of a company manufacturing disposable razors for men. In a general effort to reduce the environmental impacts of the company, you are asked to find ways to redesign the company's best selling product {\color{blue}[\stepcounter{slide}Slide \arabic{slide}]} into a more eco-efficient version. 

Form groups of three students and:
\begin{enumerate}
	\item Define the product's functional unit. 
	\item Using a whiteboard or a large piece of paper on which you draw the MECO matrix, identify the environmental impacts involved in the product life cycle. Since we don't really have access to detailed product information, don't hesitate to make assumptions about material qualities and quantities processed in the product life cycle.
	\item Identify the hotspots of environmental impacts. Here again, since we don't have enough information to do a full LCA and base our decisions on an exhaustive and quantitative evaluation of impacts, don't hesitate to follow your intuition.
	\item Using the online tool \href{http://pilot.ecodesign.at/pilot/ONLINE/ENGLISH/PDS/INDEX.HTM}{Ecodesign Pilot} {\color{blue}[\stepcounter{slide}Slide \arabic{slide}]}, look for design guidelines fitting with the hotspots you identified in the last step. The online tool will first ask you to choose between five environmental impact profiles depending on the phase where the hotspots are located. Once you have chosen one profile you will be displayed with different categories of eco-design strategies. Browse through them to see which of those you could apply to your product. Read the guidelines headlines and the explanatory text. More explanatory text is available by hoovering over the label ``learn''. By applying these guidelines to your product, generate and record ideas how to improve the eco-efficiency of the product. 
	\item Choose two of the more promising ideas you have recorded in the previous step and generate a conceptual design.
	\item Finally, present your conceptual design to your colleagues. 
\end{enumerate}
}

\section*{Credits}
\label{sec:credits}
%%%%%%%%%%%%%%%%%%%%%%%%%%%%%%%%%%%%%%%%%%%%%%%%%%%%%%%%%%%%%%%%%%%%%%%%%%%%%%%%%%%%%%%%%%%%%%%%%%%%
%%%%%%%%%%%%%%%%%%%%%%%%%%%%%%%%%%%%%%%%%%%%%%%%%%%%%%%%%%%%%%%%%%%%%%%%%%%%%%%%%%%%%%%%%%%%%%%%%%%%
These works are released under a \href{https://creativecommons.org/licenses/by/4.0/}{Creative Commons Attribution 4.0 International License}.

\bibliographystyle{ieeetr}
\bibliography{../References}
\end{document}
