\documentclass{article}

% style elements
%   Slide numbers {\color{blue}[\stepcounter{slide}Slide \arabic{slide}]}
%   Key readings \textsuperscript{\color{Magenta}[key reading]}
%   For work-in-progress snippets {\color{red}|some text|}

\usepackage[utf8]{inputenc}
\usepackage{titling}
\usepackage{hyperref}
\usepackage{color}
\usepackage[dvipsnames]{xcolor}
\usepackage{comment}
\usepackage{textcomp}

\newcounter{slide}
\newcounter{exercise}

\newcommand{\subtitle}[1]{%
  \posttitle{%
    \par\end{center}
    \begin{center}\large#1\end{center}
    \vskip0.5em}%
}

\begin{document}

\title{Future trends in NPD - Part I\\Citizen participation in product development}

\author{Product Design and Development (ME30294). Lecture 11. \\ Jérémy Bonvoisin, Dept. Mech. Eng., University of Bath}
\date{Last update: \today}

\maketitle

\begin{abstract}
The ever-growing affordability of manufacturing and information technologies combined with raising criticism about the consumption society led to the emergence of a ``maker culture'' promoting citizens' rights to keep control on everyday technologies. Formerly pushed in the background as a passive consumer, today's citizen enjoys an increasing capacity to take action and to participate in product development. This lecture provides an overview of this contemporary evolution and sketches the future of NPD as a process involving institutions and individual people collaborating in distributed communities to create shared designs.

{{\it Keywords.} open innovation; crowdsourcing; open source hardware.}
\end{abstract}

\tableofcontents

%%%%%%%%%%%%%%%%%%%%%%%%%%%%%%%%%%%%%%%%%%%%%%%%%%%%%%%%%%%%%%%%%%%%%%%%%%%%%%%%%%%%%%%%%%%%%%%%%%%%
%%%%%%%%%%%%%%%%%%%%%%%%%%%%%%%%%%%%%%%%%%%%%%%%%%%%%%%%%%%%%%%%%%%%%%%%%%%%%%%%%%%%%%%%%%%%%%%%%%%%
\section{Design for everyone}
\label{sec:context}
The vision delivered in this lecture is those of an increasing participation of individual people in product development, within and outside the firm. This vision is at the confluence of three contemporary trends: the ever increasing demand for customized products, the increased accessibility of design tools for laypeople and the emergence of counter-cultures challenging the distribution of roles in the production of everyday products. 

\subsection{A short history of manufacturing}
\label{sec:history}
It is nowadays commonplace to observe that since mid 20\textsuperscript{th} century manufacturing follows a trend towards lower production volumes per model and higher product variety. In the first half of the century, the industry strove to escape \emph{craft production} and targeted at \emph{mass production}---its efforts were focused on increasing volumes and lowering variety in order to harvest economies of scale. The mindset of the mass manufacturing era is well exemplified by the citation attibuted to Henry Ford: ``Any customer can have a car painted any colour that he wants so long as it is black''\footnote{I don't know whether Henry ford \emph{actually} said this but this is not important here. What is important is that this presumed citation is well known in industry and is symptomatic for the mindset of this era.}. Since the second half of the century, industries increasingly strove to take advantages from new technologies in order to come back to the starting point {\color{blue}[\stepcounter{slide}Slide \arabic{slide}]}. It strove to offer the advantages of craft production in terms of customization together with those of mass production in terms of price. This led us to the current era of \emph{mass customization}. Nowadays, the number of possible variants offered by the automotive industry goes far beyond the number of actually produced cars. ``In 2004, Daimler Chrysler produced about 1.1 million Mercedes A class at the production plant in Raststatt. Only two of these cars were completely identical.'' \cite{piller2010handbook}.

This evolution towards more customization is predicted to lead to a new paradigm of `personal fabrication' where ``customers create innovative products and realize value by collaborating with manufacturers and other consumers'' \cite{huEvolvingParadigmsManufacturing2013a}\textsuperscript{\color{Magenta}[key reading]}. In this new paradigm, the role of the customer is not only to \emph{buy} like in the mass manufacturing paradigm or to \emph{choose} like in the mass customization paradigm: it is also to \emph{design} their own products {\color{blue}[\stepcounter{slide}Slide \arabic{slide}]}. The role of the industry also changes to become those of a facilitator of a 'commons-based peer-production': ``a model of social production, emerging alongside contract- and market-based, managerial-firm based and state-based production. These forms of production are typified by two core characteristics. The first is decentralization. Authority to act resides with individual agents faced with opportunities for action, rather than in the hands of a central organizer, like the manager of a firm or a bureaucrat. The second is that they use social cues and motivations, rather than prices or commands, to motivate and coordinate the action of participating agents'' \cite{benklerCommonsbasedPeerProduction2006}. 

\subsection{Democratization of design}
\label{sec:democratizationofdesign}
The emergence of personal fabrication is made conceivable by the radical increase in the range of design and production tools accessible to laypeople in the last decade {\color{blue}[\stepcounter{slide}Slides \arabic{slide}-\addtocounter{slide}{5}\arabic{slide}]}. It is now common for pupils or students to have access to a desktop-size machine-tools such as a 3D-printer or a CNC mill. Medium-size machines such as laser cutters are increasingly available to every interested person in a growing number of FabLabs and Makerspaces. At the same time, there is a growing offer of easy to use 3D-modelling software such as \href{https://www.tinkercad.com/}{TinkerCAD} as well as of open source or free-to-use parametric CAD software such as \href{https://www.freecadweb.org/}{FreeCAD} or \href{https://www.onshape.com/}{OnShape}. It is easier than ever to share 3D-models with the use of online libraries such as \href{https://www.thingiverse.com/}{Thingiverse} or in-browser 3D-viewers such as \href{https://sketchfab.com/}{Sketchfab}. There is also an increasing offer of \href{https://en.wikipedia.org/wiki/Computer-supported_cooperative_work}{computer supported collaborative work (CSCW)} platforms such as \href{https://github.com/}{GitHub} allowing to structure coordinated work and manage shared content. 

Tools are more accessible, so do the possibilities to learn how to use them. Know-how is easily shared through video tutorials or wikis {\color{blue}[\stepcounter{slide}Slide \arabic{slide}]}. It is easily exchanged between peers in community forums, in hackathons or in makerspaces. The combined accessibility of tools and know-how comes along with an increased self-confidence to tackle activities related to product design, prototyping and production. It tends to `relieve the need for formal segmentation' \cite{chenDirectDigitalManufacturing2015a} between designers, producers and users. It challenges the traditional distinction between professional and layperson and establishes the figure of the normal citizen as a qualified and credible stakeholder of product development.

\subsection{Maker culture}
\label{sec:Maker culture}
This evolution is contemporary to the emergence of the `\href{https://en.wikipedia.org/wiki/DIY_ethic}{do-it-yourself}', the `\href{https://en.wikipedia.org/wiki/Hacker_culture}{hacker}' and `\href{https://en.wikipedia.org/wiki/Maker_culture}{maker}' subcultures {\color{blue}[\stepcounter{slide}Slide \arabic{slide}]}. They tend to reject consumerism and passive dependence on established social structures (such as schools, corporations, governments) {\color{blue}[\stepcounter{slide}Slide \arabic{slide}]}. Instead, they promote self-empowerment, active learning (learning-by-doing), 'do-it-yourself' and the vision of a society where people are valued for what they are able to do instead of what they are able to buy. They value the figure of a `prosumer' taking action to influence their consumption patterns. They also tend to value voluntary and collective effort as well as unconditional knowledge sharing.

The capacity of these subcultures to exceed the field of spare-time activities and to become mainstream, that is, to disrupt production patterns outside some niche markets, is disputable {\color{blue}[\stepcounter{slide}Slide \arabic{slide}]}. Nonetheless, they tend to challenge the traditional distinction between professional and layperson and to establish the figure of the normal citizen as a qualified and credible stakeholder of product development. They are the visible part of the iceberg and personify a larger contemporary evolution towards increased citizen participation in product design. 

\begin{comment}
In the following sections, we will review two trends taking advantage of the creative force of individual citizens in NPD processes:
\begin{itemize}
	\item \emph{Crowdsourcing} is a firm-centered concept and is about outsourcing part of the product development process to the public in order to reduce R\&D costs, generate more original product ideas or improve the company's image. 
	\item \emph{Open Source Hardware} is a community-centered concept and is about developing freely replicable products in open and collaborative processes, using the same principles than open source software. 
\end{itemize}
\end{comment}

% end of section - time for take aways and exercise
{\color{PineGreen}
\setlength{\parskip}{1em}
{\emph{Take-aways of this section:
\setlength{\parskip}{0em}
\begin{itemize}
  \item The 50 last years of history of manufacturing shows a strong trend towards product customization and customer participation.
  \item Laypeople have increased capacity to act as product designers, hence blurring the border between the figures of the professional designer and the hobbyist.
	\item There is a cultural change from the citizen as a passive consumer to the new figure of the `maker' or `prosumer'.
\end{itemize}
}}
}

%%%%%%%%%%%%%%%%%%%%%%%%%%%%%%%%%%%%%%%%%%%%%%%%%%%%%%%%%%%%%%%%%%%%%%%%%%%%%%%%%%%%%%%%%%%%%%%%%%%%
%%%%%%%%%%%%%%%%%%%%%%%%%%%%%%%%%%%%%%%%%%%%%%%%%%%%%%%%%%%%%%%%%%%%%%%%%%%%%%%%%%%%%%%%%%%%%%%%%%%%
\section{Participation in firm-led product development}
\label{sec:firmcentered}

Today's best practices of innovation management encourage permeability of the firm's boundaries for inbound and outbound information flows (see section \ref{sec:OI}). In line with this, companies show an increased interest in sourcing innovative product ideas from the `crowd' (see section \ref{sec:crowdsourcing}).

\subsection{Open Innovation}
\label{sec:OI}

``Today, the common understanding of the innovation process builds on the observation that firms rarely innovate alone and that innovation is a result of interactive relationships among producers, users, and many other different institutions [...]. The early Schumpeterian (1942) model of the lone entrepreneur bringing innovations to markets has been superseded by a richer picture of different actors in networks and communities'' \cite{pillerSocialMediaSocial2011}. This contemporary understanding replaces the formerly dominant ``funnel model'' of innovation based on a stage-gate product development process {\color{blue}[\stepcounter{slide}Slide \arabic{slide}]} where ``the initial phase is a wide screening of the raw ideas in order to find the most successful one; then, the approved ideas are turned into projects and developed; of these, just few are actually launched on the market'' \cite{colomboOpenInnovationMeets2016}. The new dominant model of innovation is those of an `\emph{open innovation}'*, defined as ``the use of purposive inflows and outflows of knowledge to accelerate internal innovation, and expand the markets for external use of innovation, respectively'' \cite{chesbroughOpenInnovationNew2006}{\color{blue}[\stepcounter{slide}Slide \arabic{slide}]}. ``While in purely closed innovation discarded ideas or projects do not generate value, in the open innovation paradigm, they can be licensed to external companies as intellectual property, or generate spin-offs and enter new markets. [... External] sources of knowledge can be leveraged in to order to foster novel ideas at the beginning of the NPD process or during the process as technology insource'' \cite{colomboOpenInnovationMeets2016}. In other words, while the closed innovation model ``involves keeping all of a company’s information private and protected, open innovation encourages making varying amounts of proprietary information public to allow a larger sampling of people to participate in solving the problem.'' \cite{petersonSocialProductDevelopment2014}. 

\subsection{Crowdsourcing and related concepts}
\label{sec:crowdsourcing}

In line with this, there is nowadays an increased interest in opening the product development process to the participation of external people, regardless of their affiliation to any institution. This increased interest is revealed by the large variety of partly overlapping names given to this phenomenon: co-creation, crowdsourcing, mass collaboration, user innovation, lead-user methods, mass collaboration, cloud-based design, social product development, etc. {\color{blue}[\stepcounter{slide}Slide \arabic{slide}]}. Co-creation is for example defined as an ``active, creative and social collaboration process between producers and customers in the context of new product development'' \cite{pillerSocialMediaSocial2011}. It is about involving a specific type of people: the users, customers, people who already know the product or have interest in using it. As for \emph{crowdsourcing}*, it is defined as ``the act of a company or institution taking a function once performed by employees and outsourcing it to an undefined (and generally large) network of people in the form of an open call'' \cite{howeCrowdsourcingDefinition2006}. The concept of crowdsourcing is not specific to product development but can apply to this type of corporate activies. And here, in contrary to co-creation, the people involved are neither necessarily users nor they need to have any interest in using the product. 

\subsection{Examples of crowdsourcing}
\label{sec:CSexamples}

The French sportswear company \textbf{\href{https://team.fr.raidlight.com/categories/atelier-de-conception-ouvert-a-tous.369/}{Raidlight}} provides a good example of co-creation {\color{blue} [\stepcounter{slide}Slide \arabic{slide}]}. They maintain a constant link with their community of users, organize design challenges, let users vote for the best ideas, and invite users to come to the R\&D department to explain their innovation ideas. Doing so, they establish the image of a participative company, they build a feeling of belonging and therewith customer fidelity, and they get innovative ideas their R\&D departments can turn into new products. They have been successful in bringing user ideas to the market and rewarding the contributors for these ideas. A trail show they issued in 2014 has the name of all contributors written on the sole {\color{blue} [\stepcounter{slide}Slide \arabic{slide}]}. 

Not only SMEs can take advantage of the creativity of the crowd, but also big players of `serious' markets such as aerospace. In 2014, \textbf{GE} launched in cooperation with the online 3D model sharing platform GrabCAD a challenge to redesign an airplane engine bracket {\color{blue} [\stepcounter{slide}Slide \arabic{slide}]}. ``The original bracket GE asked the GrabCAD Community to redesign via 3D Printing methods weighed 2.033 grams (4.48 pounds). The winner, M Arie Kurniawan, was able to slash its weight by nearly 84\% to just 327 grams (0.72 pounds)''{\color{blue} [\stepcounter{slide}Slide \arabic{slide}]}\footnote{GE jet engine bracket challenge on \href{https://grabcad.com/challenges/ge-jet-engine-bracket-challenge}{GrabCAD}}. 637 designs were submitted and the winner has been granted with a 7,000 USD award. The story does not tell whether GE actually integrated the bracket into any jet engine. Nonetheless, ``\$7,000 for that lighter bracket seems like a pittance compared to its economic impact, or even what GE would have paid a high-caliber engineer to solve the problem.''\footnote{Elisabeth Stinson, How GE plans to act like a startup and crowdsource breakthrough ideas, \href{https://www.wired.com/2014/04/how-ge-plans-to-act-like-a-startup-and-crowdsource-great-ideas/}{wired magazine}, 04.11.2014}. Contrarily to Raidlight who pays back contributors with symbolic rewards (e.g. public acknowledgments), rewards are here pecuniary and based on the ``winner takes it all'' principle.

\textbf{\href{https://quirky.com}{Quirky}} {\color{blue}[\stepcounter{slide}Slide \arabic{slide}]} is one of the most cited examples of successful crowdsourcing because its entire business is based on bringing ideas from the crowd to the market. They help individual creators bringing products to the market through a 4-step online process: 1) the creator submits an idea 2) the idea is reviewed by the community of other registered creators, 3) Quirky takes over the detailed design and industrialization to bring the product to the market and 4) pays royalties to the creator. The company claims to have a community of more than 1m users and to have  paid back more than 10m USD royalties to the creators. Contrarily to the GE's challenge, contributors do not compete but collaborate and reward is conditional to the extent and the success of ones contributions. 
\begin{comment} 
https://www.emeraldinsight.com/doi/pdfplus/10.1108/JBS-10-2016-0120
\end{comment}

Companies like \textbf{\href{https://www.openideo.com/}{OPENIdeo}} {\color{blue}[\stepcounter{slide}Slide \arabic{slide}]} or \href{https://www.innocentive.com/}{Innocentive} implement the same kind of process but as a service for other companies. They involve individual people in challenges initiated by client companies. These challenges often focus wider issues than on product design (for example: \href{https://www.innocentive.com/ar/challenge/9934083}{Seeking an Alternative to Raw Wood for Packaging and Fastening of Cargoes in Sea Transportation} or \href{https://www.openideo.com/challenge-briefs/nike-design-with-grind}{How might we create a waste-free, circular future by designing everyday products using Nike Grind materials?}).

% end of section - time for take aways and exercise
{\color{PineGreen}
\setlength{\parskip}{1em}
{\emph{Take-aways of this section:
\setlength{\parskip}{0em}
\begin{itemize}
  \item The business world shows an increasing interest to exchange product-related information outside the boundaries of the firm.
  \item There is a wide variety of practices involving individual people as active stakeolders of product development.
	\item Among these, crowdsourcing even acknowledges the competence of individual players as designers (or engineers).
\end{itemize}
}}
}


%%%%%%%%%%%%%%%%%%%%%%%%%%%%%%%%%%%%%%%%%%%%%%%%%%%%%%%%%%%%%%%%%%%%%%%%%%%%%%%%%%%%%%%%%%%%%%%%%%%%
%%%%%%%%%%%%%%%%%%%%%%%%%%%%%%%%%%%%%%%%%%%%%%%%%%%%%%%%%%%%%%%%%%%%%%%%%%%%%%%%%%%%%%%%%%%%%%%%%%%%
\section{Community-based product development}
\label{sec:community}
Not only can individual people and communities participate online to company-led product development processes, but product development processes can also happen outside the field of influence of the firm. There is an increasing number of examples of products designed in open processes involving no control from institutional stakeholders and where any interested volunteer is welcomed to participate. These processes require the free revelation of product-related information: they require the product to be `open source' so any interested person can participate (see section \ref{sec:OSH}). This and the reliance on voluntary work makes these processes hardly compatible with existing management techniques and raises exciting challenges for new NPD models (see section \ref{sec:challengesOD}).

\subsection{Open Source Hardware}
\label{sec:OSH}
\emph{Open Source Hardware}* (OSH) products are those products ``whose design has been released to the public in such a way that anyone can make, modify, distribute, and use'' them \cite{opensourcehardwareassociationOpenSourceHardware2016} {\color{blue}[\stepcounter{slide}Slide \arabic{slide}]}. In other words, an OSH product is a physical artefact whose documentation is released under a license granting anyone with production and distribution rights, and is detailed enough to enable anyone to study and develop it further. 

The concept of OSH is based on an adaptation of the four freedoms of open source (first stated in the context of software development by the free software fundation \cite{freesoftwarefoundationFreeSoftwareDefinition2015}) to the field of physical products \cite{powellDemocratizingProductionOpen2012}: 
\begin{itemize}
	\item \textbf{Hardware freedom 0}. The freedom to use the device for any purpose.
	\item \textbf{Hardware freedom 1}. The freedom to study how the device works and change it to make it to do what you wish---access to the complete design is precondition to this.
	\item \textbf{Hardware freedom 2}. The freedom to redistribute the device and/or design (remanufacture).
	\item \textbf{Hardware freedom 3}. The freedom to improve the device and/or design, and release your improvements (and modified versions in general) to the public, so that the whole community benefits.
\end{itemize}

The four freedoms of OSH require the publication of the `source' of a `design'. In contrast with software, there is no unique definition of what the `source' of a hardware product is \cite{bonvoisinWhatSourceOpen2017}. Instead, it may take different forms depending on the intention of the product originator to engage into OSH {\color{blue}[\stepcounter{slide}Slide \arabic{slide}]}:
\begin{itemize}
  \item The originator either wants 1) to build transparency in order to be trusted by their customers or 2) to support product maintainability, repairability or upgradability. In these cases, relevant documentation may be computer aided design (CAD) files and drawings.
  \item The originator wants the product to be widely produced and adopted beyond their own sphere of influence, either for charitable reasons or because they can indirectly profit from a wide adoption. In this case, relevant documentation may be bills of materials and assembly instructions.
  \item The originator intends to create a community-based product development process allowing the participation of any interested developer. In this case, it is not only relevant to share the CAD files, but also information about the development process: what are the expected requirements for the product, what has been already achieved and what are the pending tasks, how can a contributor join in the project, etc.
\end{itemize}

\subsection{History and contemporary relevance}
\label{sec:OSHhistory}
\paragraph{From software to hardware.} OSH results from a recent extension of the open source movement outside the domain of software and into the realms of physical products {\color{blue}[\stepcounter{slide}Slide \arabic{slide}]}. In software engineering, open source products have been developed for around 30 years and generate nowadays billion-dollar businesses. The first domain to which the principles of open source have been extended is electronic hardware. The flagship of this new era and one of today's most successful companies building on open source hardware is \href{https://www.arduino.cc/}{Arduino}. More recently, the extension of the open source principles has also reached other types of physical products such as mechanical products, mechatronic products, construction, and textile products. Two frontrunner projects which raised large attention from the public are \href{https://localmotors.com/}{Local Motors}\footnote{Local Motors's business model however moved over time from open-source to crowdsourcing.} and \href{https://www.opensourceecology.org/}{Open Source Ecology}. 

\paragraph{OSH as a grassroots movement.} For now, OSH products have mainly been issued by grassroots communities, non-commercial sectors or freelance businesses. A typical example of OSH product born in academia is \href{https://reprap.org}{RepRap} {\color{blue}[\stepcounter{slide}Slide \arabic{slide}]} which has been adopted by a vivid community of enthusiasts, scholars and enterprises, who generated an impressive number of \href{https://reprap.org/wiki/RepRap_Family_Tree}{remixes}. A famous example of grassroots community built upon OSH principles is \href{https://www.opensourceecology.org/}{Open Source Ecology} {\color{blue}[\stepcounter{slide}Slide \arabic{slide}]}, a project aiming at developing and building a `Global Village Construction Set', i.e. a set of 50 open source industrial machines allowing to `build a small civilization with modern comfort'. Some of their machines---for example the brick press---have been replicated several times outside the initial community. An example of product resulting from a freelance business is \href{http://www.hovalabs.com/hova-instruments/hovalin}{Hovalin} {\color{blue}[\stepcounter{slide}Slide \arabic{slide}]}, ``a functional acoustic violin that can be produced using most standard consumer 3d printers''. The 3D-models of the product are freely available and are provided with detailed assembly instructions. For those who don't want to build their violin by themselves, the authors offer to sell either the fully assembled products or the parts to be assembled.

\paragraph{OSH in businesses.} OSH has also found its way to marketplaces {\color{blue}[\stepcounter{slide}Slide \arabic{slide}]}. Some emerging businesses such as start-ups and medium-size enterprises have built their operations on OSH, especially in the supply market for makers. Two examples of them are the companies \href{https://ultimaker.com/}{Ultimaker} and \href{https://www.alephobjects.com/}{Aleph Objects}, both developing, manufacturing and distributing 3D-printers. Another example is the company \href{https://ztautomations.com/openbeam/}{OpenBeam}, producing extruded aluminum framing systems for rapid prototyping of machinery building. OSH also raised interest of larger industrial players as exemplified by the automotive industry. Tesla, one of the newcomers and challengers in this field, engaged in this way by \href{https://www.tesla.com/blog/all-our-patent-are-belong-you}{declaring they would ``not initiate patent lawsuits against anyone who, in good faith, wants to use our technology.''} The more established French automobile company Renault announced in 2016 a \href{https://www.openmotors.co/renaultpomsignup/}{partnership with Open Motors to open up the electric car Twizy}. Although the realization of these strategies has not issued concrete open source content so far, they indicate that OSH gained attention out of the sphere of grassroots initiatives and non-monetary economies.

\subsection{Development process models in OSH}
\label{sec:dvppractices}

OSH is primarily an intellectual property management approach departing from the dominant logic of \href{https://en.wikipedia.org/wiki/Intellectual_property}{intellectual property protection}. At first glance, this has nothing to do with NPD. At first glance only, because OSH is also a tool for connecting with a community of people around product and its development. With this regards, OSH is also an enabler for an alternative family of open product development models---three of which are introduced hereafter.

\paragraph{Evolutionary design.}How product development in OSH looks like first depends on the complexity of the product. On one side of the spectrum, there is the publication of simple products designs performed by individual ``home engineers'' in online sharing places for CAD models such as \href{http://thingiverse.com/}{Thingiverse}, which counts more than 1.1m uploaded objects. Although these objects are generally designed as one-person-projects, they are part of collective development practices in the form of sequential series of remakes: one maker develops one version which is taken over and developed further by someone else, and so on \cite{kyriakouKnowledgeReuseCustomization2017}{\color{blue}[\stepcounter{slide}Slide \arabic{slide}]}. This model can be termed as \emph{evolutionary design} in the sense that designs evolve in generations like species adapt generation after generation to their environment. 

\paragraph{Public innovation.}On the other side of the spectrum, there are complex products---those which combine different technologies, are made of several parts and designed to satisfy demanding needs. The workload associated with their development outweigh the capacity of single individuals. It must be performed in teams, that is, in collaborative development processes. There are basically two approaches to the development of complex OSH products in collaborative development processes {\color{blue}[\stepcounter{slide}Slide \arabic{slide}]}. The first approach is to reveal the result of a product development project performed in a private setting, what is called \emph{public innovation}. The the end of this process is marked by the publication/revelation of the product documentation which has been kept private so far. This case corresponds to conventional NPD as described in this module, excepted that the result is made public. 

\paragraph{Open design.}The second approach is to develop the product in a community-based setting, what we term \emph{open design}\footnote{The term open design is not very specific and may be used in practice or in scientific literature to describe only partly related things. We use this term here in the sense of `Open Source Product Development', as described \href{https://opensourcedesign.cc/wiki/index.php/Open_Source_Product_Development}{here}}. In this case, the process is open to the participation of external people and the end of the process is marked by the release of already public documents in a stable version. This approach to product development is the most promising but as well the most challenging in OSH. Its major characteristics and challenges are sketched in the next subsection.

\subsection{Challenges of open design}
\label{sec:challengesOD}
Open design differs in various ways from conventional NPD. Hereafter a list of a few characteristics colliding with conventional management techniques and making of open design a challenging field.
\begin{itemize}
  \item Participation of external contributors is on a voluntary basis. There is no formal obligation of result or of effort and no direct retribution for the job done. Nothing more than self-motivation and moral engagement forces a contributor to finish their job in a reasonable delay or to finish it at all. However, contributions may be indirectly rewarded by things like the good feeling of fulfilled duty, access to the collaboratively designed product, peer recognition or increased influence in the design process, among others {\color{blue}[\stepcounter{slide}Slide \arabic{slide}]}. In this context, project initiators have interest in publicizing what kinds of indirect rewards eventual contributors may draw from their efforts.
	\item In the absence of formal employment, there is no lever for enforcing any pyramid-shaped hierarchy. The organizational structure rather resembles a mix of a do-o-cracy and meritocracy. The person who takes the initiative has the word and decides how to proceed (do-o-cracy). The more a person does, the more they gather recognition from their peers and can weigh in discussions (meritocracy). Nonetheless, hierarchies based on file access privileges can be maintained using \href{https://en.wikipedia.org/wiki/Distributed_version_control}{distributed version control systems} like \href{https://git-scm.com/}{Git} do {\color{blue}[\stepcounter{slide}Slide \arabic{slide}]}.
	\item The absence of formal employment is also an absence of formal engagement. People can pop in and out as they will. This may result in a high personal turnover and make it challenging to maintain continuity in the project. This aspect reinforces the need to work in openly accessible files and to document the process. Clear process and overview of the current state of design make it easy for newcomers to jump in an to take over tasks where they have been left out by others. In this context, project initiators have an interest in promoting transparency, in providing the corresponding shared IT tools and keeping documentation up to date {\color{blue}[\stepcounter{slide}Slide \arabic{slide}]}.
	\item Division of work is based on self-selection of tasks. The absence of any hierarchy enforcement mechanism makes it impossible for a central authority to assign tasks. Instead, contributors may take themselves initiatives, identify by themselves what needs to be done based on the information they have and pick the tasks they are more keen on addressing. In this context, an option to keep control on the direction taken by the product development is to maintain a curated list of pending tasks in a central `to-do-list' or an issue tracking system contributors can pick tasks from {\color{blue}[\stepcounter{slide}Slide \arabic{slide}]}.
	\item There is generally an important activity imbalance within communities. As a rule of thumb, the distribution of workload among contributors follows a hyperbolic shape {\color{blue}[\stepcounter{slide}Slide \arabic{slide}]}: 90\% of the development is usually done by a core team of devoted members, 9\% may be performed by a surrounding community of sporadic contributors and the remaining 1\% may be done by a larger community of anecdotic contributors. In this context, project initiators may face the dilemma to spend their time in development tasks or in stimulating the collaborative activity in order to correct this natural imbalance.	
	\item The progress of product development does not only depend on the volume of incoming contributions, but also on how far these contributions make the process converge towards a satisfactory design. In order to support quality and convergence, it is common practice that all contributions are submitted to a review panel made of influential contributors having a good overview of the product. However, there is always the possibility that a subgroup of the community disagrees with this panel, with the core team or more generally with the direction taken by the development process. In that case, they always have the possibility to branch-out and start a new project. The phenomenon of branching out is well illustrated by the \href{https://makezine.com/2015/12/02/a-reprap-family-tree-tracking-the-printers-that-started-it-all/}{RepRap family tree} {\color{blue}[\stepcounter{slide}Slide \arabic{slide}]} showing a central development branch with satellite branches based on alternative product visions.
\end{itemize}

\bigskip

This depicts open design as hardly compatible with the requirements of a firm-centered product development process. Indeed, it may not be relevant for a company to use open design to outsource a product development process whose success and time line would be critical for the company. It may be more interesting to consider open design as an opportunity to let emerge products out of a process performed in an ecosystem of people and organizations. An iconic example of this organization of work is delivered by Linux whose development is shared between employees of a large number of companies (80\% of contributions) and volunteers (20\% of contributions) \cite{corbet2012linux} {\color{blue}[\stepcounter{slide}Slide \arabic{slide}]}. While it is established that such development practices work for software, there hasn't been any open source hardware development project of this size yet. The question remains open: what will be the Linux of hardware?

% end of section - time for take aways and exercise
{\color{PineGreen}
\setlength{\parskip}{1em}
{\emph{Take-aways of this section:
\setlength{\parskip}{0em}
\begin{itemize}
  \item Product design not is an exclusivity of the firm: it can be initiated and performed in other forms of organization, including informal communities.
	\item While business is fuelled by intellectual property protection, community-based product development is fueled by open source.
	\item Since open design (as a sub-type of community-based product development) relies on voluntary work, it cannot be managed like firm-led product development---interesting organizational challenges are at work. 
  \item While open design is still in limbo and is rather an emerging concept than a well established practice, it gives an idea of what the future of design could be.
\end{itemize}
}}
}


%%%%%%%%%%%%%%%%%%%%%%%%%%%%%%%%%%%%%%%%%%%%%%%%%%%%%%%%%%%%%%%%%%%%%%%%%%%%%%%%%%%%%%%%%%%%%%%%%%%%
%%%%%%%%%%%%%%%%%%%%%%%%%%%%%%%%%%%%%%%%%%%%%%%%%%%%%%%%%%%%%%%%%%%%%%%%%%%%%%%%%%%%%%%%%%%%%%%%%%%%
\section{Outlook: tomorrow's designer}
\label{sec:designerofthefuture}
The increasing capacity of the individual citizen to take over design-related tasks challenges the design activity as the exclusive remit of a profession. If everyone can be a designer, what is the role of a professional designer? What is the specific expertise which would still make the difference between a professional designer and other people? 

\paragraph{Thesis 1.}If R\&D departments get more and more involved in crowdsourcing and open design, the role of the designer (and the engineer) may switch to the those of a `design manager'. They would less \emph{design} than \emph{let good design emerge}. No matter how good the work of external people may be, there will always be a risk in terms of quality. The role of the trained designer would then be to review this work, to point at good design practice and ask community members ``have you thought about this aspect in your design?''.

\paragraph{Thesis 2.}Independent from this, design practice may get closer to some kind of ``google engineering'' already prevalent in software development. Designers and engineers could increasingly take advantage of components libraries to `shop' partial solutions to their design brief (on the model of software libraries). Solution finding could increasingly happen online, making it possible for other designers to retrace the collective cognitive process and recall ideas they could use (on the example of programmer Q\&A forums like \href{https://en.wikipedia.org/wiki/Stack_Overflow}{Stack Overflow}).

\begin{comment}
{\color{red}Die OSPE ist einerseits gekennzeichnet durch eine evolutionäre Entwicklung von OSH-Produkten auf Basis unabhängig voneinander geleisteter Weiterentwicklungen und andererseits der Koordination und Integration verteilter Entwicklungsbeiträge. Unterstützende Software (wie z.B. kollaborative CAD-Software) und eine modulare Produktarchitektur senken den Aufwand für manuelle Integration auf ein möglichst geringes Niveau. Die Zusammenführung von Beiträgen aus dem verteilten Entwicklungsprozess zur Veröffentlichung einer offiziellen Produktversion erfolgt zentral durch einen festgelegten Zuständigen, der jeweils die am besten geeigneten Entwicklungslösungen auswählt und integriert. > products are modular (even maybe open architecture products), designer is integrator}
\end{comment}

\section*{Credits}
\label{sec:credits}
%%%%%%%%%%%%%%%%%%%%%%%%%%%%%%%%%%%%%%%%%%%%%%%%%%%%%%%%%%%%%%%%%%%%%%%%%%%%%%%%%%%%%%%%%%%%%%%%%%%%
%%%%%%%%%%%%%%%%%%%%%%%%%%%%%%%%%%%%%%%%%%%%%%%%%%%%%%%%%%%%%%%%%%%%%%%%%%%%%%%%%%%%%%%%%%%%%%%%%%%%
These works are released under a \href{https://creativecommons.org/licenses/by/4.0/}{Creative Commons Attribution 4.0 International License}. 

\begin{comment}
https://www.researchgate.net/publication/320601945_Distributed_economies_through_open_design_and_digital_manufacturing
\end{comment}

\bibliographystyle{ieeetr}
\bibliography{../References}

\begin{comment}
{\color{gray}\section*{Appendix: four advantages of open source}
\label{sec:advantages}
The relevance of open source and especially of OSH is supported by sustainability, business and macroeconomic arguments.

\paragraph{Product quality.} The most cited argument in favor of open source is that ``given enough eyeballs all bugs are shallow''. In other words, the more people can have a look, the more issues they will be able to raise and solve. Individuals can check whatever they value as quality and eventually take action to improve the product. This can for example lead to safer products, as discussed in the software branch, where open source has been advocated to lead to better digital security. This can also lead to more durable products: potential breaking points can be more easily identified and product obsolescence openly discussed.

\paragraph{Citizen capability.} Open source goes along with a lower user dependence in case of product fault. Having access to technical information allows users to find solutions by themselves in case the product originator cannot provide those, whatever the reason. This supports users to extend product lifetime for economical or environmental reasons, if they wish so. A side effect is to support technological literacy. Open source provides people wanting to develop their technical skills with more cases to quench their thirst for learning or doing things on their own.

\paragraph{R\&D efficiency.} Opening the product development process to the participation of volunteers is a promise of either decreased R\&D costs or increase of fresh and innovative ideas coming in. Letting other people participate enables the emergence of an original ecosystem of stakeholders the company may not have thought working with in the first place. It may also allow identifying key talents to be hired. In summary information disclosure may contribute to the prosperity of the firm throuhg better, cheaper products. 

\paragraph{Speed of innovation.} Openness supports reusability of intellectual assets. It helps avoiding ``reinventing the wheel'' and accelerating product development. By having access to the technical documentation of other products you can pick up the things you need in them. This is what massively happens in software development and why it is so fast nowadays. Because you just find on the internet the bits of code you need in 99\% of the cases. Moreover, the replicability of hardware may support faster adoption of best available technologies. Inventions are not locked-in but made available, which increases the chances they will actually be provided where they are needed.}
\end{comment}

\end{document}

