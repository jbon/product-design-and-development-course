\documentclass{article}

\usepackage[utf8]{inputenc}
\usepackage{titling}
\usepackage{hyperref}
\usepackage{color}
\usepackage[dvipsnames]{xcolor}
\usepackage{comment}

\newcounter{slide}
\newcounter{exercise}

\newcommand{\subtitle}[1]{%
  \posttitle{%
    \par\end{center}
    \begin{center}\large#1\end{center}
    \vskip0.5em}%
}

\begin{document}

\title{Sustainability and eco-design}
%\subtitle{Course Product Design and Development. Lecture 8.}
\author{Jérémy Bonvoisin, Dept. Mech. Eng., University of Bath}
\date{Last update: July 2018}

\maketitle

\begin{abstract}
This lecture discusses the implications of PDD in terms of environmental sustainability and gives an overview of the corresponding mitigation measures referred under the umbrella of eco-design---a term defined as the maximisation of the ratio between the product functionality and the associated environmental impacts. In a first section, the lecture recalls the general notions of sustainability and environmental impact and links these general concepts to the design-specific concepts of product life cycle and functional unit. The second section introduces the rationale of eco-design and provides an overview of some eco-design strategies. The last section introduces some of the available tools for eco-design implementation in practice. 
\end{abstract}

\section{Products and sustainability}
\label{sec:sustainability}

According to the most widely accepted definition, \emph{sustainable development} or \emph{sustainability}* ``is development that meets the needs of the present without compromising the ability of future generations to meet their own needs'' \cite{brundtland1987our}. The ``ability of future generation to meet their own needs'' is generally understood as depending from the preservation of three types of capital: the environmental\footnote{in other words: the `nature', the `natural resources', including living biological stock, usable mineral resources, breathable air and drinkable water, beautiful landscapes, etc.}, social\footnote{in other words: human well-being, happiness, and its constituents being eventually, health, education, equality of chances, freedom, etc.}, and economical capitals\footnote{in other words: money and other forms of marketable assets.}. 

Further efforts to break down this programmatic definition into concrete terms is inevitably bound to political implications and is therefore subject to diverging interpretations. One main topic of disagreement is whether capitals are interchangeable, that is, for example, whether a certain amount of social capital can be sacrificed for a bit of environmental capital. In the one hand, proponents of the so-called \emph{weak sustainability} consider the three capitals as interchangeable and of equal importance and seeks for a preservation or growth of their total balance. This school of thought tends to consider the environment from an utilitarian point of view. That is, the environment is 1) the supplier of all the resources we need to satisfy human needs and 2) the repository of all we don't need anymore. Following this idea, the `function' of natural cycles is to absorb wastes and to turn them into restored resources. Caring about the environment is a matter of ensuring the further fulfillment of human basic needs (e.g. breathe, eat, drink, reproduce) and well-being (to which may contribute things like mediated communication, mechanised transportation, cultural creation, entertainment, comfort, representation). On the other hand, proponents of the so-called \emph{strong sustainability} consider there should be no substitution. This school of thought tends to consider the environment from an idealistic perspective and to speak for nature rights. It assumes that preservation of the social capital is the objective of human activity; those of the economical capital is a mean to achieve it; those of the environmental capital is understood as a condition. Weak sustainability is generally represented as a Venn diagram and strong sustainability as concentric circles {\color{blue}[\stepcounter{slide}Slide \arabic{slide}]}. Adhesion to one or the other school of thought is ultimately a question of \emph{Weltanschauung}, one's world view, their own philosophy.

Without taking position in this debate, we can state that the sustainability of human activities relates to the extent to which they:
\begin{itemize}
	\item create social value (they are \emph{useful} to someone). Social value is assumed to be positive and is to be maximized. 
	\item contributes to environmental degradation (they have \emph{environmental impacts*}). Environmental degradation is assumed to be negative and is to be minimized.
	\item participate to the creation of economical value added (they can be charged for money). From the point of view of the firm, economical value added is considered as positive and is to be maximized to ensure viability. The weak and strong sustainability disagree on whether economical value added has to be maximized from a macroeconomic point of view.
\end{itemize}

Consequently, in the context of this lecture, we consider that caring for sustainability consists in maximising social value and minimising the associated environmental impacts while contributing to economical viability. {\color{red}In the following, we further consider that the economical and social aspects of sustainability are already understood and therefore focus on explaining the environmental dimension.}

\subsection{Environmental impacts}
\label{sec:EnvironmentalImpacts}

Let's approach the concept of environmental impact with a simple analogy: suppose you want---whatever the reason---to dig a hole in your well-grassed garden {\color{blue}[\stepcounter{slide}Slide \arabic{slide}]}. By doing it, you reduce the available surface to dig another hole in your garden, that is, you consume a specific resource which is the ``area of nicely grassed ground". This resource becomes more scarce than it was before. Digging a hole also inevitably creates a heap somewhere else, further reducing the resource ``area of nicely grassed ground" of an additional amount (the surface of the heap base). 

In this analogy, the hole stands for the \emph{consumption}*, the heap stands for the \emph{pollution}* and both together stand for the \emph{depletion of a resource}* (the garden area). Resource depletion refers to a decrease in the available stock of resources---to the destruction of a bit of what we called earlier the ``environmental capital''. Environmental impact is the marginal contribution of a process to this depletion due to the direct consumption of resources or through the emission substances leading to its pollution. 

This stated, let's now have a closer look at the kind of resources can be depleted by consumption and what kind of pollution can contribute to this depletion.

\subsubsection{Resource consumption, depletion and restoration}
\label{sec:depletion}
Consumption can either take the form of a destruction of the resource or its dispersion to an extent which hinders further use. An example of dispersion is the use of precious metals in semiconductors: extracting the ores, refining the metals, embedding them in products and using these products does not obliterate the chemical elements, it just leads to their geographic dispersion which ultimately makes them unavailable for further industrial use. An example of resource destruction is the combustion of oil-based fuels: the resource disappears as it is transformed into something else (CO\textsubscript{2}, among others). Resources ongoing destructive processes may further either be renewable or non-renewable. Destructed renewable resources can be restored, either through natural cycles (like clean freshwater) or through human intervention (like food). In this case, depletion happens when the destruction of resources is quicker than the ability of the restoring mechanism to replenish the stock. The destruction of non-renewable resources can neither be restored by natural cycles nor by human intervention. Once a species is extinguished, once nuclear or fossil fuel is burnt, there is no comeback.

Excessive consumption is already or is about to become a critical issue for a large variety of economic sectors {\color{blue}[\stepcounter{slide}Slide \arabic{slide}]}. The remaining reserves of some precious metals like gallium and arsenic (used in semiconductors), indium and silver (used in photovoltaic panels), as well as gold and silver (used in electronic circuits) may be depleted in 5 to 50 years if consumption and disposal continues at present rate \cite{dodsonElementalSustainabilityTotal2012}. Those of uranium (used for nuclear energy production) as well as cadmium and nickel (used in batteries) may be depleted within 50 to 100 years (\emph{ibid.}). The depletion of other resources like oil, coal or wild fish is a well-known issue often covered in the media. The depletion of phosphate rock is another less covered ongoing issue threatening agriculture worldwide \cite{cooperFutureDistributionProduction2011}.

\subsubsection{Pollution}
\label{sec:pollution}
Pollution is defined as the introduction of physical (e.g. radiation, chemical substances) or biological (e.g. manure, seeds) agents into an ecosystem to an amount it cannot be absorbed, hence leading to its adverse disruption {\color{blue}[\stepcounter{slide}Slide \arabic{slide}]}. Pollution can be either natural (like the disastrous \href{https://en.wikipedia.org/wiki/1980_eruption_of_Mount_St._Helens}{eruption of Mount St. Helens} in 1980 and reducing hundreds of square miles to wasteland) or anthropogenic (like the ``\href{https://en.wikipedia.org/wiki/Smog}{smog}" affecting the health of citizen in urban environments). It can either be accidental (like the \href{https://en.wikipedia.org/wiki/Fukushima_Daiichi_nuclear_disaster}{Fukushima nuclear disaster}) or chronic (like the smog, once again). Its causes can either be isolated (like in the case of the \href{https://en.wikipedia.org/wiki/Hinkley_groundwater_contamination}{Hinkley groundwater contamination}) or dispersed (like the \href{https://en.wikipedia.org/wiki/Marine_debris}{omnipresence of long-lasting waste in oceans}). Among the most often considered the anthropogenic, chronic and dispersed pollutions are {\color{blue}[\stepcounter{slide}Slide \arabic{slide}]}:
\begin{itemize} % add examples or news for each of the examples (mer de plastic en espagne, protocole de montreal...)
	\item \href{https://en.wikipedia.org/wiki/Eutrophication}{eutrophication}: excessive growth of algae or plants in a body of water resulting from the excessive intake of nutrients such as nitrates or phosphates.
	\item \href{https://en.wikipedia.org/wiki/Acid_rain}{acid rains}: unusual acidity of rainfalls due to emissions of sulfur dioxide and nitrogen oxyde in the air
	\item \href{https://en.wikipedia.org/wiki/Ozone_depletion}{ozone hole}: depletion of the stratospheric ozone layer at Earth's poles due to the emission of ozone depleting substances such as CFCs, leading to a reduced share of UV radiation dispersed by the ozone layer. 
	\item \href{https://en.wikipedia.org/wiki/Smog}{smog}: noxious smoky fog hitting dense urban areas, composed of airborne particles and chemical compounds generated by combustion and leading to respiratory issues. 
	\item \href{https://en.wikipedia.org/wiki/Global_warming}{global warming}: rise in the average temperature of the Earth's surface resulting from an increased atmospheric greenhouse effect due to the massive airborne release of greenhouse gases such as carbon dioxide and or methane
	\item \href{https://en.wikipedia.org/wiki/Light_pollution}{light pollution}: high concentration of misdirected light at night in densely populated areas which may cause diffuse and chronic adverse effects on human health and disrupts wild life (e.g. bird migration) 
	\item \href{https://en.wikipedia.org/wiki/Land_use}{land use}: anthropogenic transformation of the natural terrestrial environment into functional spaces, leading to various adverse effects such as reduction of biodiversity, soil erosion, and water quality reduction. 
\end{itemize}
These examples show that there isn't \emph{one} pollution but rather a variety of ways ecosystems can be affected by anthropogenic activities. Different effects may even influence each other and be produced by the same agents, like the ozone depletion and global warming having complex interrelations and both being influenced by CFC emissions. Pollution may also affect complex sets of resources in a way that is not always understood, like global warming not only affecting weather conditions, but fresh water stock, also emmerged land, biological stock etc...

\subsubsection{Contextual aspects}
\label{sec:context}
Unfortunately, there is no global reference of resources that need to be preserved from depletion. One reason is the absence of agreement on what is to be considered as an important resource. Buddhist people would have directly thought about the well-being of bugs and worms in the garden analogy, other wouldn't. Opinions may differ whether the well-being of bugs is important, whether the available garden area is more important than the well-being of bugs, or whether these things are important at all. This also illustrates that there is no consensus on the relative importance of resources. Is it worse to contribute to ozone layer depletion or to the disruption of bee colonies? {\color{blue}[\stepcounter{slide}Slide \arabic{slide}]} All this is the expression of ethical values underpinning different positions in the debate between strong and weak sustainability. Another reason is the partly unequal global distribution of resources, making them differently critical for local populations: some regions have enough fresh water not to care about it, some countries have enough oil not to care about energy. A third reason is the absence of agreement on mechanisms to allocate critical resources, which is an utterly political topic. 

There is no global reference of pollutions that need to be mitigated as well. One reason is that some of them are localized, such as smog and light pollution only affecting urban areas. Whether pollution is considered as detrimental also depends on the perceived criticality of the resources involved in the affected ecosystem---and we have seen that there is no global consensus on that. Nevertheless, there are examples of measures to mitigate specific pollutions at global level, such as the \href{https://en.wikipedia.org/wiki/United_Nations_Framework_Convention_on_Climate_Change}{United Nations Framework Convention on Climate Change}, or the \href{https://en.wikipedia.org/wiki/Montreal_Protocol}{Montreal Protocol on Substances that Deplete the Ozone Layer}. A larger number and variety of agreements can be found at regional level, like the legally binding \href{https://en.wikipedia.org/wiki/European_emission_standards}{European emission standards for new vehicles} or more local levels. 

Consequently, which environmental impacts needs to be considered is in relation with a given geographical or political context. This makes it difficult to find comparable references to measure environmental impacts. 

\subsubsection{About irreversibility, or why less is better}
\label{sec:irreversibility}
Thermodynamic delivers us nice concepts to bypass this difficulty and to find simpler concepts to lean upon. In thermodynamical terms, depletion of resources can be termed as creation of \emph{entropy}. And the second law of thermodynamics tells us an interesting thing about entropy: in a closed system, it inescapably increases.

Let's apply this to the garden analogy. Suppose now you don't need the hole anymore and want to put everything back in place. You can restore the previous situation by putting the heap back in the hole. The result will however only \emph{approximate} the original situation: you will still have a discontinuity in the grass where the hole was and eventually a damaged grass where the heap was. Restoring the soil and grass in \emph{exactly} the same condition would require way more time and energy than what was necessary to dig the hole (e.g. you will eventually need to grow new grass). Because, in thermodynamical terms, what you want to do is to remove from the system the entropy you added by digging the hole. Removing entropy from a closed system requires creating elsewhere more entropy than the amount you want to remove. The domain of water management delivers another illustration of this principle of irreversibility {\color{blue}[\stepcounter{slide}Slide \arabic{slide}]}: mixing freshwater with any kind of liquid pollutant is easy and barely requires energy. But reversing the process, that is, separating the liquids, is difficult and may requires a lot of energy. What thermodynamics says, in other words: the energy required to mess up something is lower than those required to tidy it up.

Another interpretation of the second law of thermodynamics is that creating order somewhere is inescapably bond to creating more disorder somewhere else. The more organized you want the matter to get, the more disorder you have to create somewhere else. An illustration of this is provided by the semiconductor industry: the production of a 2g microchip, which embeds a fairly high organisation of matter requiring high material purity, requires more than 1.7kg of matter \cite{williamsKilogramMicrochipEnergy2002}. That is, only 1.1\% of all matter involved in the production process ends up in the final product, 98.9\% turns out to be waste. Creating a platinum ring requires refining approx. 100kg of ores \cite{erkmanVersEcologieIndustrielle2004}, building a car 70t \cite{janinDemarcheEcoconceptionEntreprise2000}, sending an SMS 600g \cite{federico2001mips}. Creating fancy things creates huge messes. 

Understanding environmental impact as entropy amounts to say that every activity has an environmental impact, which is irreversible. Trying to restore some resources (to ``recycle'' them) would only make the whole set of resources worse. Consequently, the best way to avoid environmental impact is to reduce activity. Less is better. In other words, the more social value is created out of the minimum activity, the better. Sustainability is about reducing waste, understood in the sense of the consumption of a resource which is not bound to human satisfaction.

% end of paragraph - time for take aways and exercise
\setlength{\parskip}{1em}
{\emph{Take-aways of this paragraph:
\setlength{\parskip}{0em}
\begin{itemize}
  \item The environmental issue considered in sustainability is resource depletion
	\item Resources can either be depleted by consumption or by pollution
	\item The importance of a resource depends on contextual factors
	\item The environmental impact of a process is a marginal contribution to resource depletion
	\item Every activity has an environmental impact
	\item less is better
\end{itemize}
}}

\setlength{\parskip}{1em}
{\color{PineGreen}\emph{Exercise \stepcounter{exercise}\arabic{exercise}. In groups of 4 people, discuss which of the environmental impacts you consider as important, how your views may fit with your working environment and the local conditions you are living or working in.}}
\setlength{\parskip}{0em}


\subsection{The environmental impacts of products}
\label{sec:tbd}
Obviously, single products neither generate massive pollution or deplete by themselves. It is the combination of small environmental impacts generated by a large amount of diverse products (like all cars or all energy using home appliance) which generates pollution. Products don't even create impacts by themselves, impacts come rather from the sum of all activities which contributed to the delivery of its functionality. The set of all these activities is called the \emph{Product lifecycle}*.

\subsubsection{Product lifecycle}
\label{sec:plc}


\begin{comment}
\subsubsection{Measure environmental sustainability}
\label{sec:localvsglobal}
As a consequence, it is not possible to say \emph{whether} an activity is sustainable or not in absolute terms. There is no such a thing as a threshold objectively marking the limit between a sustainable and an unsustainable activity. The best thing we can do is to state sustainability in relative, comparative terms. (because if you set thresholds, you are prescribing a allocation mechanism, which is a utterly political thing). The EU made it for Co2 emissions of cars, setting an absolute threshold. 

added contributions of small impacts create polution and depletion

\subsubsection{Functional unit}
\label{sec:fu}

\subsubsection{the role of product design}
\label{sec:productdesign}

20\%-80\%-rule
0
\section{ecodesign rationale}
\label{sec:ecodesignRationale}
 - ratonal approach
 - 4 levels of Brezet
 - radical change vs. end-of-pipe 

\section{tools for implementation}
\label{sec:tools}
ecodesign pilot - communication - engineering tools


\section{Take-aways}
\label{sec:TakeAways}
\begin{itemize}
	\item product lifecycle
	\item Folie 19 AIIT
\end{itemize}

\end{comment}

\bibliographystyle{plain}
\bibliography{references}
\end{document}
